\section{Motivation}
The volume of knowledge found on the web is growing considerably, so the interest of many 
communities is in profiting from that knowledge. Many questions are being asked  
to search engines like Google, which serves roughly 4.2 million searches done by users every 
minute\footnote{\href{https://www.internetlivestats.com/}{https://www.internetlivestats.com/}}. 
Since most of the data found on the web does not have a standard structure, 
search engines do not tend to reply to the question directly but just to retrieve the documents 
that might contain the answer. Though many questions can be answered by doing so, many 
other more complex questions require a higher level of reasoning that is difficult to achieve 
by consulting only unstructured data. Thus, there is still a challenging problem with making 
data more accessible, even knowing that its volume is increasing exponentially.
% \href{https://www.internetlivestats.com/}{Internet Live Stats - Internet Usage \& Social Media Statistics}

To give semantic meaning to all this information available on the Web in a manner in which both 
humans and machines can understand, a common framework is required. Thus, 
the Semantic Web~\cite{key:semwebsa} was proposed as an extension of the World Wide Web built on 
standards set by the World Wide Web Consortium (W3C). The primary purpose of this 
initiative is to support a “Web of Data” where data can be searched like in databases, but 
at the scope of the Web. The compilation of Semantic Web techniques and tools provides 
a framework where applications can query that data, draw inferences using vocabulary, etc. 
Thus, the ultimate goal is to extend the variety of tasks that computational systems can 
support, while developing trusted interactions over the network. 

The Semantic Web establishes a standard method to describe data using the Resource 
Description Framework (RDF)~\cite{key:rdfprimer11}. This data model describes resources using statements 
of the form subject-predicate-object, also called triples, and can be represented as a directed 
edge-labelled graph. A collection of RDF statements is known as a knowledge graph (KG)~\cite{key:ldbook}. 
Altogether, these KGs when linked together on the Web give shape to what is called 
the Linked Data Cloud~\cite{key:ldprinciples}: a large amount of interlinked RDF datasets that comprise more 
than 30 billion RDF triples. Among the most popular KGs, Wikidata~\cite{KG:wikidata} and DBpedia~\cite{KG:dbpedia} 
are huge and become more useful and accessible each day for research fields and applications~\cite{wikidata:usage-MalyshevKGGB18, EL:dbpedia-spotlight-MendesJGB11}. 

Wikidata~\cite{KG:wikidata} is a free open knowledge graph that can be read and edited by both humans 
and machines. Since the Wikimedia Foundation first launched Wikidata in October 2012, 
it has grown vastly. It has served as a reference resource for many of Wikimedia’s sister 
projects, like Wikipedia, the largest virtual encyclopedia on the Web. Wikidata is a valuable 
and comprehensive source of knowledge. It is supported mainly by its community and is 
designed in a way that people from all over the world can contribute. Many applications have 
used Wikidata as an information provider such as Apple’s Siri; it has also been used for research activities in life 
sciences and social science, and it even is used by Google to empower its search engine~\cite{wikidata:usage-MalyshevKGGB18}.

Thereafter, for querying this vast amount of data available on the web, a query language is 
needed. SPARQL~\cite{key:sparql11} is a query language able to retrieve and manipulate data stored in RDF 
format. The main advantages of SPARQL are that it allows for writing queries that follow RDF 
specifications and provides a specific graph transversal syntax for querying arbitrary-length 
paths in graphs.

While all of this knowledge available in the public domain drives growing interest in doing 
research regarding the Semantic Web, there is also a need to have a basic understanding of 
how data is structured (RDF) and how to access this data (SPARQL). These requirements 
represent a barrier-to-entry for non-expert users. Hence these barriers lead to the broad and 
complex challenge of developing intuitive and easy-to-use interfaces for end-users.

Many solutions have emerged to approach this issue, among which natural language interfaces 
such as Question Answering Systems (QAS)~\cite{qa:survey-BOUZIANE2015366, qa:intro-UngerFC14, qa:nn-qakg-Chakraborty19} 
have been receiving much attention. 
These systems aim to answer questions posed by humans in natural language, extracting the 
answer from one or more sources. There are QASs able to retrieve answers from an unstructured 
collection of natural language~\cite{qa:survey-BOUZIANE2015366}. However, we are interested in systems able to 
construct their responses by querying structured data, like relational databases or knowledge 
graphs from the Linked Data Cloud.

More specifically, the task of answering natural language questions using knowledge graphs 
is known as Question Answering over Knowledge Graphs (QAKG)~\cite{qa:nn-qakg-Chakraborty19} or Question 
Answering over Linked Data (QALD)~\cite{qa:intro-UngerFC14, qa:qald-Lopezetal2013}. Unger et al.~\cite{qa:intro-UngerFC14} define the QALD task
as follows: \textit{“translate the users’ information need into a form such that they can be evaluated using 
standard Semantic Web query processing and inference techniques.”} Their work also describes 
the types of questions these systems aim to answer, which often focus on definition questions 
(“Who was Tom Jobim?”) or factoid questions. These last ones that can be divided into 
predicative questions (“Who was the first man in space?”), list questions (“Give me all cities 
in Germany”) or boolean questions (“Was Margaret Thatcher a chemist?”). 

Several Question Answering systems have been developed to address some of the main 
challenges in Question Answering, commonly using pattern matching~\cite{qa:pattern-FaderZE13, qa:pattern-LopezFMS12}, 
grammar-based techniques~\cite{qa:grammar-DamljanovicAC10, qa:grammar-2-Marginean17}, 
or graph exploration~\cite{qa:graph-XuFZ14, qa:graph-2-ZouHWYHZ14} approaches. Although these systems have 
shown good results when dealing with a considerable amount of questions, their performance 
decreases~\cite{qa:challenges-semweb-HoffnerWMULN17} with questions involving more complex graph patterns or with vocabulary 
mismatches caused by the user typing different terms to the ones contained in the knowledge 
graph from which the information is being retrieved (this issue is also known as the lexical gap~\cite{semPar:lexical-gap-HakimovUWC15}). One example 
is the question \textit{“Which US player is the highest scorer in World Cups?”} which could require 
complex operations like aggregation and sorting (count goals, sort and retrieve the maximum 
scorer) and might have vocabulary mismatches (it is not made explicit that we refer to FIFA World 
Cups, or the US might not be registered as an abbreviation of United States).

Aside from works that have tried to mitigate these problems~\cite{semPar:lexical-gap-HakimovUWC15, semPar:complex-queries-GliozzoK12}, some approaches 
that rely on Semantic Parsing have shown positive results due to recent advances in Deep 
Learning applied to Natural Language Processing~\cite{semPar:sempar-as-mt-AndreasVC13}. Semantic Parsing is the process of 
mapping a natural language sentence into a formal representation of its meaning~\cite{semPar:sempar-as-mt-AndreasVC13}. Some 
applications include code generation~\cite{semPar:code-gen-RabinovichSK17, semPar:tranx-code-gen-YinN18}, 
automated reasoning~\cite{semPar:ITPKaliszykUV17} or query construction~\cite{semPar:txt-to-sql-RadevKZZFRS18}. 
In particular, Andreas et al.~\cite{semPar:sempar-as-mt-AndreasVC13} discussed how Semantic Parsing could benefit from using 
Machine Translation methods, whereby natural language is “translated” into a structured 
representation. Following their work, some Neural Machine Translation (NMT) approaches 
have brought about a growing interest in applying deep neural networks to Semantic Parsing 
problems~\cite{nmt:CaiXZYLL18, nmt:DongL16, nmt:ZhongCoRR17}. In NMT, pairs of sequences are given as input to a Deep Neural Network 
model, which is expected to learn the translation model. A natural idea is then to try to apply 
the NMT approach for translating Natural Language (NL) to SPARQL and some works have begun to explore such techniques~\cite{nmt:CoRRLuz18, nmt:nspm-SoruMMPVEN17, nmt:CoRRSoru18}.

There are multiple challenges relating to converting a NL question to its SPARQL query 
representation (NL-to-SPARQL); some of these challenges are directly inherited from the 
original NL-to-NL translation problem, while others are distinct. First, there isn’t a one-to-one 
mapping for every NL question to a SPARQL query. On one hand, there are multiple ways to 
express a question in NL. For example, the question \textit{“How far away is the Earth from the Sun?”} 
can also be rephrased as \textit{“What is the distance between the Sun and Earth?”}. On the other 
hand, questions can be translated to SPARQL in different ways. For example, a question \textit{“What 
is the largest country in Africa?”} might be translated to a query based on population or area, 
where one such translation must be chosen, and where both give different answers. Moreover, 
one SPARQL query has potentially equivalent queries that will return the same results (over any 
data). In the question about US players, for example, we can first retrieve US players and then 
count their scores, or vice versa; thus there is a need to establish some common conventions 
when designing NL-to-SPARQL systems. 

Another issue is the lack of corpora for training NL-to-SPARQL models when compared to 
the enormous amount of documents in different languages that can be found on the Web for 
training NL-to-NL models (e.g. news, blogs, articles, academic documents, etc.). Generating 
data for NL-to-SPARQL is not an easy task considering the need for a basic SPARQL understanding 
to build such datasets, although there is some work regarding automating parts 
of the process of generating NL-to-SPARQL pairs~\cite{dataset:dbnqa-hartmann-marx-soru-2018, dataset:lcquad-TrivediMDL17}. 
Furthermore, the queries required to answer a question change from dataset to dataset, where the SPARQL queries needed for 
DBpedia are not the same as those for Wikidata.

One of the most recent works in regards to using NMT for SPARQL is that 
the Neural SPARQL Machine (NSpM)~\cite{nmt:nspm-SoruMMPVEN17}, which considers SPARQL as a foreign language. 
The main idea is to train an end-to-end learning model to translate any NL expression 
into a sequence of tokens in the SPARQL grammar that expresses a query equivalent to the 
question over a given dataset. Question Answering systems based on neural networks usually 
aim to generate the whole SPARQL query in an attempt to perform the entire process of 
identifying the relevant entities along with deducing the KB properties, triples, and operators 
that would retrieve the expected answer. 

An analysis of the performance of many NMT models on translating NL to SPARQL 
has been performed by Yin et al.~\cite{nmt:nl-to-sparql-Yin19}, where eight deep neural network models were tested 
over different datasets based on questions over DBpedia. One relevant dataset is the \textit{Large 
Complex Question Answering Dataset} (LC-QuAD)~\cite{dataset:lcquad-TrivediMDL17}, consisting of 5,000 complex questions 
based on 38 hand-made templates. Another dataset is the \textit{DBpedia Neural Question Answering} 
dataset (DBNQA)~\cite{dataset:dbnqa-hartmann-marx-soru-2018} that includes almost 900,000 questions based on templates extracted 
from questions of the LC-QuAD dataset and the 7\textsuperscript{th} version of the \textit{Question Answering over 
Linked Data} dataset (QALD-7)~\cite{dataset:qald7-UsbeckNHKRN17}.

Though the performance of these models shows promising potential for constructing meaningful 
and useful SPARQL queries, they present some key limitations relating to the data 
used to train the models. First, despite the fact that many NMT models report positive results 
when evaluated over simple and large datasets like the DBNQA dataset~\cite{nmt:nl-to-sparql-Yin19}, which 
contains questions with little variation in syntax and phrasing, such regular data do not give an accurate 
understanding of the real performance of these systems. In fact, the performance of such models drops 
dramatically when evaluated over more complex questions like the ones included in the LC-QuAD 
dataset, which might not contain enough questions to learn accurately~\cite{nmt:nl-to-sparql-Yin19}. Second, 
the current models are vocabulary-dependent, which means there is no capability for recognizing 
new entities or properties that were not used in the training data~\cite{nmt:nl-to-sparql-Yin19}. These issues lead 
to the constant need to train the model with new, manually created pairs of NL questions 
and SPARQL queries.

The first issue can be addressed by building a more comprehensive dataset that has enough 
cases for an NMT to learn properly while maintaining an abstraction level that allows us to 
respond accurately to complex and diverse questions. Regardless, creating new datasets does 
not necessarily help with the vocabulary dependency issue, unless we have examples using 
every entity and property in the knowledge graph, which seems infeasible to generate in the 
short-to-medium term. Therefore, an important the goal is to maximize the available training examples, 
where there is a need to find an alternative approach to complement the parsing power 
of NMTs with a system that helps to address vocabulary dependency by extracting the 
information NMTs cannot generalize.

For example, NMTs cannot be expected to extract entities from a question and translate them to their identifiers in the KG. 
Developing labelled examples for each entity does not seem feasible, as mentioned before. 
Plenty of solutions have addressed the entity extraction task over Linked Data. In particular, 
the Information Extraction area, which involves the automatic extraction of implicit 
information from unstructured or semi-structured sources, intersects in many ways with the 
Semantic Web~\cite{infExtr:MartinezHL19}. Some examples are systems that perform Named Entity 
Recognition~\cite{ner:LampleBSKD16}, Sequence Labeling~\cite{seqlab:MaH16, seqlab:contextual-emb-AkbikBV18}, 
or Entity Linking~\cite{EL:dbpedia-spotlight-MendesJGB11, EL:aida-tool-YosefHBSW11, EL:tagme-FerraginaS10, EL:opentapioca-Delpeuch19} 
while leveraging Semantic Web resources and/or techniques. 
In particular, Entity Linking (EL) systems aim to perform the entire process of identifying entity names in a 
text, mapping names to KB resources, and disambiguating them depending on the context 
of a given corpus~\cite{EL:survey-WuHH18}. Considering again the question \textit{“Which US player is the highest scorer 
in World Cups?”}, an EL system could effectively identify the resources associated with the 
country US or the FIFA World Cup. Many EL systems~\cite{EL:dbpedia-spotlight-MendesJGB11, EL:aida-tool-YosefHBSW11, EL:tagme-FerraginaS10, EL:opentapioca-Delpeuch19} 
have achieved positive results when linking KB entities over text and these systems tend to generalize well 
over any new corpus. 

Furthermore, these Information Extraction tools can be complemented along with intermediate 
representation of structured queries. An example can be found on a proposed 
Text-to-SQL system~\cite{semPar:txt-to-sql-RadevKZZFRS18}, where given a question in NL, an intermediate representation of 
an SQL query is generated, consisting of a SQL template with slots to be filled later with 
relevant words identified in the question using Named Entity Recognition tools~\cite{ner:dynet-NeubigDGMAABCCC17}.

We see an opportunity to improve state-of-the-art performance for QASs in the context 
of RDF/SPARQL by exploring the idea of combining the parsing capacity of NMT to get an 
intermediate representation of a SPARQL query, with the entity extraction and disambiguation 
power of Entity Linking systems to identify the relevant entities in questions, decreasing 
the dependency of current QA systems on training examples that cover the full vocabulary of a 
knowledge graph. Additionally, there are many challenges to address – as we have previously mentioned 
– like how to deal with different representations of the same question (e.g. paraphrased questions), 
what canonical representation of SPARQL queries we should adopt, how to evaluate 
that a QAS is fulfilling its purposes, among others.
