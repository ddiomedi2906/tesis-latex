\chapter{Question Answering Dataset}
\label{appendix:qaDataset}
We provide more details on the data used for the training and validation of our system.

\section{Normalized Dataset Format}
\label{appendix:qaDataset/normalizedFormat}
As mentioned in the \textit{Experimental Design} chapter~\ref{cap4:experimentalDesign}, we proposed 
a dataset format so the process 
of generating the Query Template dataset and the Sequence Labeling datasets could be simplified. This 
normalized format is also used for the other test datasets (QALD-7 and WikiSPARQL). In 
Listing~\ref{lst:normalizedDatasetExample} we can see an example of what information each case contains:

\begin{sparqlcode}[%
    caption={Example of one \LCQuADtwo{} case following our proposed normalized format.}, 
    label={lst:normalizedDatasetExample}]
    "question_id": 30226,
    "natural_language_question": "Did Alexander Hamilton practice law?",
    "query_answer": [
        {
            "query_id": 0,
            "sparql_query": 
                "ASK WHERE { wd:Q178903 wdt:P106 wd:Q40348 }",
            "entities": [ 
                {"label": "Alexander Hamilton", "entity": "wd:Q178903"},
                {"label": "lawyer", "entity": "wd:Q40348"}
            ],
            "slots": [
                {"slot": "<sbj_1>", "label": "Alexander Hamilton"},
                {"slot": "<obj_1>", "label": "lawyer"}
            ],
            "sparql_template": "ASK WHERE { <sbj_1> wdt:P106 <obj_1> }"
        }
    ]
\end{sparqlcode}

\begin{itemize}
    \item \textbf{Question ID}: the unique identifier of the question; for \LCQuADtwo{} we decided to 
    treat each verbalized and paraphrased version of the normalized question as a separate case. 
    Verbalized cases are then numbered from $0$ to $30,225$, and paraphrased cases from $30,226$ to 
    $60,451$.
    \item \textbf{Natural Language Question}: the question text (verbalized or paraphrased version).
    \item \textbf{Query Answer}: a list of possible SPARQL query valid responses. We allow more than 
    one possible query since for each question there are many ways to reach the expected answer (for 
    example, \dquotesit{What is the biggest country?} may refer to population, area, etc.; however in 
    this work we only have one SPARQL query per case). Each query answer case has the following fields:
    \begin{itemize}
        \item \textbf{Query ID}: unique identifier to identify query answers for the same question.
        \item \textbf{SPARQL query}: the SPARQL query string.
        \item \textbf{Entities}: list of expected annotations of the entities being used in the 
        SPARQL query answer. Each annotation contains the entity URL and the label of the question 
        associated with that entity.
        \item \textbf{Slots}: list of expected slots to use for the Slot Filling system. Each slot 
        contains the associated label in the question and its corresponding placeholder in the Query 
        Template.
        \item \textbf{SPARQL Template}: the Query Template derived from the SPARQL query.
    \end{itemize}
\end{itemize}

\section{LC-QuAD 2 base templates}
\label{appendix:qaDataset/baseTemplates}
In the \textit{Results} chapter~\ref{cap5:results} we displayed an analysis based on the 22 base 
templates used for building 
the \LCQuADtwo{} dataset~\cite{dataset:lcquad2-DubeyBA019}. According to each base template structure, 
we established which cases are considered \textbf{complex cases}, which allows us to estimate the 
percentage of complex questions this dataset contains. A base template is considered a \textit{complex 
case} if it includes any of the following operations: (1) counting, (2) filtering, (3) ordering, 
(4) use of strings or numbers,(5) use of property statements, or (6) returns more than one variable. 

Next, we present a brief overview of each one of these base templates along with a representative 
example from the dataset:

\begin{enumerate}
    \item \textbf{ask\_one\_fact}: ASK query with one query triple.
    \begin{itemize}
        \item Example: \textit{Did Alexander Hamilton practice law?}
        \item Query:\\
        \mbox{}\\
        \begin{sparqlcode}[]
ASK WHERE { 
    wd:Q178903 wdt:P106 wd:Q40348 
}
        \end{sparqlcode}
    \end{itemize}

    \item \textbf{ask\_one\_fact\_with\_filter}: ASK query with one query triple and a numeric filter 
    operation (either less than, equal, or greater than).
    \begin{itemize}
        \item Example: \textit{Does the standard molar entropy of silver equal 34.08?}
        \item Query:\\
        \mbox{}\\
        \begin{sparqlcode}[]
ASK WHERE { 
    wd:Q1090 wdt:P3071 ?obj filter(?obj = 34.08) 
}
        \end{sparqlcode}
    \end{itemize}

    \item \textbf{ask\_two\_facts}: ASK query with two query triples. Note that the same subject 
    and property are used in both query triples.
    \begin{itemize}
        \item Example: \textit{Was William Henry Harrison both a United States senator and Governor of 
        Indiana?}
        \item Query:\\
        \mbox{}\\
        \begin{sparqlcode}[]
ASK WHERE { 
    wd:Q11869 wdt:P39 wd:Q16147601 . 
    wd:Q11869 wdt:P39 wd:Q13217683 
}
        \end{sparqlcode}
    \end{itemize}

    \item \textbf{count\_one\_fact\_object}: SELECT query with COUNT operation from one query triple. 
    Count the number of objects that match the query triple.
    \begin{itemize}
        \item Example: \textit{How many Latin conjugations are there?}
        \item Query:\\
        \mbox{}\\
        \begin{sparqlcode}[]
SELECT (COUNT(?obj) AS ?value ) WHERE { 
    wd:Q397 wdt:P5206 ?obj 
}
        \end{sparqlcode}
    \end{itemize}

    \item \textbf{count\_one\_fact\_subject}: SELECT query with COUNT operation from one query triple. 
    Count the number of subjects that match the query triple.
    \begin{itemize}
        \item Example: \textit{What is the number of spore print colors for olive?}
        \item Query:\\
        \mbox{}\\
        \begin{sparqlcode}[]
SELECT (COUNT(?sbj) AS ?value ) WHERE { 
    ?sbj wdt:P787 wd:Q864152 
}
        \end{sparqlcode}
    \end{itemize}

    \item \textbf{rank\_instance\_of\_type\_one\_fact}: SELECT query with two query triples with 
    sorting and limit. The first query triple always uses the property instance of (wdt:P31). 
    Ordering might vary (ascending or descending).
    \begin{itemize}
        \item Example: \textit{What battery power station has the highest amount of energy storage 
        capacity?}
        \item Query:\\
        \mbox{}\\
        \begin{sparqlcode}[]
SELECT ?ent WHERE { 
    ?ent wdt:P31 wd:Q810924 . 
    ?ent wdt:P4140 ?obj 
} ORDER BY DESC(?obj) LIMIT 5
        \end{sparqlcode}
    \end{itemize}
    
    \item \textbf{rank\_max\_instance\_of\_type\_two\_facts}: SELECT query with three query triples using 
    sorting and limit. The first query triple always uses the property instance of (wdt:P31). 
    Descending ordering is used to get the maximum value.
    \begin{itemize}
        \item Example: \textit{What is the largest village in Muchinigi Puttu?}
        \item Query:\\
        \mbox{}\\
        \begin{sparqlcode}[]
SELECT ?ent WHERE { 
    ?ent wdt:P31 wd:Q532 . 
    ?ent wdt:P2046 ?obj . 
    ?ent wdt:P131 wd:Q11107378 
} ORDER BY DESC(?obj) LIMIT 5
        \end{sparqlcode}
    \end{itemize}
    
    \item \textbf{rank\_min\_instance\_of\_type\_two\_facts}: Same as the previous base template but with 
    ascending ordering to get the minimum value.
    \begin{itemize}
        \item Example: \textit{What is the name of a manned spacecraft in low Earth orbit with 
        smallest periapsis?}
        \item Query:\\
        \mbox{}\\
        \begin{sparqlcode}[]
SELECT ?ent WHERE { 
    ?ent wdt:P31 wd:Q7217761 . 
    ?ent wdt:P2244 ?obj . 
    ?ent wdt:P522 wd:Q663611
} ORDER BY ASC(?obj) LIMIT 5
        \end{sparqlcode}
    \end{itemize}
    
    \item \textbf{select\_object\_instance\_of\_type}: SELECT query with two query triples. The first 
    query triple always uses the property instance of (wdt:P31). Return the entities that are the 
    object of the first query triple.
    \begin{itemize}
        \item Example: \textit{What is the prefecture of Hiroshima in Japan?}
        \item Query:\\
        \mbox{}\\
        \begin{sparqlcode}[]
SELECT DISTINCT ?obj WHERE { 
    wd:Q34664 wdt:P131 ?obj . 
    ?obj wdt:P31 wd:Q50337 
}
        \end{sparqlcode}
    \end{itemize}
    
    \item \textbf{select\_object\_using\_one\_statement\_property}: SELECT query with three query 
    triples. Uses one property statement and one property qualifier. 
    \begin{itemize}
        \item Example: \textit{When did Louis XVIII of France, husband of Marie Josephine of Savoy, 
        die?}
        \item Query:\\
        \mbox{}\\
        \begin{sparqlcode}[]
SELECT ?value WHERE { 
    wd:Q7750 p:P26 ?s . 
    ?s ps:P26 wd:Q231844 . 
    ?s pq:P582 ?value
}
        \end{sparqlcode}
    \end{itemize}
    
    \item \textbf{select\_one\_fact\_object}: SELECT query with one query triple. Return the entities 
    that are the subject of that query triple.
    \begin{itemize}
        \item Example: \textit{Which are the characters for La Malinche?}
        \item Query:\\
        \mbox{}\\
        \begin{sparqlcode}[]
SELECT DISTINCT ?answer WHERE { 
    ?answer wdt:P674 wd:Q230314
}
        \end{sparqlcode}
    \end{itemize}
    
    \item \textbf{select\_one\_fact\_subject}: SELECT query with one query triple. Return the entities 
    that are the object of that query triple.
    \begin{itemize}
        \item Example: \textit{Which is the electric charge for antihydrogen?}
        \item Query:\\
        \mbox{}\\
        \begin{sparqlcode}[]
SELECT DISTINCT ?answer WHERE { 
    wd:Q216121 wdt:P2200 ?answer
}
        \end{sparqlcode}
    \end{itemize}
    
    \item \textbf{select\_one\_qualifier\_value\_and\_object\_using\_one\_statement\_property}: SELECT query 
    with three query triples. Uses one property statement and one property qualifier. Only one entity 
    used for the first query triple.
    \begin{itemize}
        \item Example: \textit{What grant did Konrad Lorenz win, and who won it with him?}
        \item Query:\\
        \mbox{}\\
        \begin{sparqlcode}[]
SELECT ?value1 ?obj WHERE { 
    wd:Q78496 p:P166 ?s . 
    ?s ps:P166 ?obj . 
    ?s pq:P1706 ?value1 . 
}
        \end{sparqlcode}
    \end{itemize}
    
    \item \textbf{select\_one\_qualifier\_value\_using\_one\_statement\_property}: SELECT query with three 
    query triples. Uses one property statement and one property qualifier. Two entities used for 
    building each query, where the second entity was used as the object of the property qualifier 
    triple.
    \begin{itemize}
        \item Example: \textit{What role did Theodore Roosevelt occupy after William McKinley?}
        \item Query:\\
        \mbox{}\\
        \begin{sparqlcode}[]
SELECT ?obj WHERE { 
    wd:Q33866 p:P39 ?s . 
    ?s ps:P39 ?obj . 
    ?s pq:P1365 wd:Q35041 
}
        \end{sparqlcode}
    \end{itemize}
    
    \item \textbf{select\_subject\_instance\_of\_type}: SELECT query with two query triples. The first 
    query triple always uses the property instance of (wdt:P31). Return the entities that are the 
    subject of the first query triple.
    \begin{itemize}
        \item Example: \textit{Which irresistible illness is caused by Staphylococcus aureus?}
        \item Query:\\
        \mbox{}\\
        \begin{sparqlcode}[]
SELECT DISTINCT ?sbj WHERE { 
    ?sbj wdt:P828 wd:Q188121 . 
    ?sbj wdt:P31 wd:Q18123741 
}
        \end{sparqlcode}
    \end{itemize}
    
    \item \textbf{select\_subject\_instance\_of\_type\_contains\_word}: SELECT query with two query triples. 
    The first query triple always uses the property instance of (wdt:P31). The second query triple 
    also includes a filter over the string value to check whether the target word is contained in 
    this string (includes filtering for the English language).
    \begin{itemize}
        \item Example: \textit{What is the title of a human that contains the word vitellius in their 
        name.}
        \item Query:\\
        \mbox{}\\
        \begin{sparqlcode}[]
SELECT DISTINCT ?sbj ?sbj_label WHERE { 
    ?sbj wdt:P31 wd:Q5 . 
    ?sbj rdfs:label ?sbj_label . 
    FILTER(CONTAINS(lcase(?sbj_label), 'vitellius')) . 
    FILTER (lang(?sbj_label) = 'en') 
} LIMIT 25
        \end{sparqlcode}
    \end{itemize}
    
    \item \textbf{select\_subject\_instance\_of\_type\_starts\_with}: SELECT query with two query triples. 
    The first query triple always uses the property instance of (wdt:P31). The second query triples 
    also includes a filtering over the string value to check whether this string starts with a target 
    letter (including filtering for the  English language).
    \begin{itemize}
        \item Example: \textit{What are the video games which start with the letter W?}
        \item Query:\\
        \mbox{}\\
        \begin{sparqlcode}[]
SELECT DISTINCT ?sbj ?sbj_label WHERE { 
    ?sbj wdt:P31 wd:Q7058673 . 
    ?sbj rdfs:label ?sbj_label . 
    FILTER(STRSTARTS(lcase(?sbj_label), 'w')) . 
    FILTER (lang(?sbj_label) = 'en') 
} LIMIT 25
        \end{sparqlcode}
    \end{itemize}
    
    \item \textbf{select\_two\_answers}: SELECT query with two query triples. Return two variables 
    (also known as double intention), with each one being the object of each query triple.
    \begin{itemize}
        \item Example: \textit{Where was Jane Austen born and where did she die?}
        \item Query:\\
        \mbox{}\\
        \begin{sparqlcode}[]
SELECT ?ans_1 ?ans_2 WHERE { 
    wd:Q36322 wdt:P19 ?ans_1 . 
    wd:Q36322 wdt:P20 ?ans_2 
}
        \end{sparqlcode}
    \end{itemize}
    
    \item \textbf{select\_two\_facts\_left\_subject}: SELECT query with two query triples.
    \begin{itemize}
        \item Example: \textit{Who were the creators of The Late Awesome Planet Soil?}
        \item Query:\\
        \mbox{}\\
        \begin{sparqlcode}[]
SELECT ?answer WHERE { 
    wd:Q22081649 wdt:P144 ?obj . 
    ?obj wdt:P50 ?answer
}
        \end{sparqlcode}
    \end{itemize}
    
    \item \textbf{select\_two\_facts\_right\_subject}: SELECT query with two query triples. Similar to 
    the previous template (in fact we could not find any difference between both cases).
    \begin{itemize}
        \item Example: \textit{What time zone is Arizona State University in?}
        \item Query:\\
        \mbox{}\\
        \begin{sparqlcode}[]
SELECT ?answer WHERE { 
    wd:Q670897 wdt:P17 ?obj . 
    ?obj wdt:P421 ?answer
}
        \end{sparqlcode}
    \end{itemize}
    
    \item \textbf{select\_two\_facts\_subject\_object}: SELECT query with two query triples. Uses one 
    entity in the subject of the first query triple, and another one in the object of the second 
    query triple.
    \begin{itemize}
        \item Example: \textit{What lake of Sao Jorge island has the tributary Curoca River?}
        \item Query:\\
        \mbox{}\\
        \begin{sparqlcode}[]
SELECT ?answer WHERE { 
    wd:Q743362 wdt:P206 ?answer . 
    ?answer wdt:P974 wd:Q10361834
}
        \end{sparqlcode}
    \end{itemize}
    
    \item \textbf{select\_two\_qualifier\_values\_using\_one\_statement\_property}: SELECT query 
    with four query triples. Uses one property statement and two property qualifiers. Return the two object 
    values of each query triple using a property qualifier.
    \begin{itemize}
        \item Example: \textit{Give me the year and name of the person with whom Bob Barker shared the 
        MTV Movie Award for Best Fight.}
        \item Query:\\
        \mbox{}\\
        \begin{sparqlcode}[]
SELECT ?value1 ?value2 WHERE { 
    wd:Q381178 p:P166 ?s . 
    ?s ps:P166 wd:Q734036 . 
    ?s pq:P585 ?value1 . 
    ?s pq:P1706 ?value2 
}
        \end{sparqlcode}
    \end{itemize}
    
\end{enumerate}