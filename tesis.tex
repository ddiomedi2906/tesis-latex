\documentclass[upright, contnum]{umemoria}

%fix for the oneside argument
\makeatletter
\g@addto@macro\titlepage{\pagenumbering{Alph}}
\g@addto@macro\endtitlepage{\pagenumbering{roman}}
\makeatother

\depto{Departamento de Ciencias de la Computación}
\author{Daniel Alejandro Diomedi Pinto}
\title{Question Answering over Wikidata using Entity Linking and Neural Semantic Parsing}
\auspicio{}
\date{Octubre 2020}
\guia{Aidan Hogan}
\carrera{Ingeniero Civil en Computación y grado de Magíster en Ciencias, Mención Computación}
\memoria{Tesis para optar al Grado de \break Magíster en Ciencias, Mención Computación \break\break Memoria para optar al titulo de Ingeniero Civil en Computación}
\comision{}

\usepackage{lipsum}

\usepackage[utf8]{inputenc}
\usepackage[T1]{fontenc}

\usepackage[toc,page]{appendix}

\usepackage{listingsutf8}
\usepackage{amsmath}  % for \hookrightarrow
\usepackage{xcolor}   % for \textcolo
\usepackage[spanish]{babel}


\lstset{
    inputencoding=utf8/latin1,
    language=SQL,
    morekeywords={PREFIX,java,rdf,rdfs,url},
    breaklines=true,
    postbreak=\mbox{\textcolor{red}{$\hookrightarrow$}\space}
}

\setcounter{tocdepth}{4}
\setcounter{secnumdepth}{4}

\begin{document}

\frontmatter
\maketitle

\begin{resumen}
    {\lipsum[1-4]}
\end{resumen}

\begin{abstract}
    {\lipsum[1-4]}
\end{abstract}

\begin{dedicatoria} % opcional
Una dedicatoria corta. Por ejemplo, al \emph{Centro Tecnológico Ucampus}
\end{dedicatoria}

\begin{thanks} % opcional
\lipsum[1-2]
\end{thanks}
\cleardoublepage

\tableofcontents
\listoftables % opcional
\listoffigures % opcional

\mainmatter
% Introduccion
\begin{intro}
	%   Motivacion
	\section{Motivation}
The volume of knowledge found on the web is growing considerably, so the interest of many 
communities is in profiting from that knowledge. Many questions are being asked  
to search engines like Google, which serves roughly 4.2 million searches done by users every 
minute\footnote{\href{https://www.internetlivestats.com/}{https://www.internetlivestats.com/}}. 
Since most of the data found on the web does not have a standard structure, 
search engines do not tend to reply to the question directly but just to retrieve the documents 
that might contain the answer. Though many questions can be answered by doing so, many 
other more complex questions require a higher level of reasoning that is difficult to achieve 
by consulting only unstructured data. Thus, there is still a challenging problem with making 
data more accessible, even knowing that its volume is increasing exponentially.
% \href{https://www.internetlivestats.com/}{Internet Live Stats - Internet Usage \& Social Media Statistics}

To give semantic meaning to all this information available on the Web in a manner in which both 
humans and machines can understand, a common framework is required. Thus, 
the Semantic Web~\cite{key:semwebsa} was proposed as an extension of the World Wide Web built on 
standards set by the World Wide Web Consortium (W3C). The primary purpose of this 
initiative is to support a “Web of Data” where data can be searched like in databases, but 
at the scope of the Web. The compilation of Semantic Web techniques and tools provides 
a framework where applications can query that data, draw inferences using vocabulary, etc. 
Thus, the ultimate goal is to extend the variety of tasks that computational systems can 
support, while developing trusted interactions over the network. 

The Semantic Web establishes a standard method to describe data using the Resource 
Description Framework (RDF)~\cite{key:rdfprimer11}. This data model describes resources using statements 
of the form subject-predicate-object, also called triples, and can be represented as a directed 
edge-labelled graph. A collection of RDF statements is known as a knowledge graph (KG)~\cite{key:ldbook}. 
Altogether, these KGs when linked together on the Web give shape to what is called 
the Linked Data Cloud~\cite{key:ldprinciples}: a large amount of interlinked RDF datasets that comprise more 
than 30 billion RDF triples. Among the most popular KGs, Wikidata~\cite{KG:wikidata} and DBpedia~\cite{KG:dbpedia} 
are huge and become more useful and accessible each day for research fields and applications~\cite{wikidata:usage-MalyshevKGGB18, EL:dbpedia-spotlight-MendesJGB11}. 

Wikidata~\cite{KG:wikidata} is a free open knowledge graph that can be read and edited by both humans 
and machines. Since the Wikimedia Foundation first launched Wikidata in October 2012, 
it has grown vastly. It has served as a reference resource for many of Wikimedia’s sister 
projects, like Wikipedia, the largest virtual encyclopedia on the Web. Wikidata is a valuable 
and comprehensive source of knowledge. It is supported mainly by its community and is 
designed in a way that people from all over the world can contribute. Many applications have 
used Wikidata as an information provider such as Apple’s Siri; it has also been used for research activities in life 
sciences and social science, and it even is used by Google to empower its search engine~\cite{wikidata:usage-MalyshevKGGB18}.

Thereafter, for querying this vast amount of data available on the web, a query language is 
needed. SPARQL~\cite{key:sparql11} is a query language able to retrieve and manipulate data stored in RDF 
format. The main advantages of SPARQL are that it allows for writing queries that follow RDF 
specifications and provides a specific graph transversal syntax for querying arbitrary-length 
paths in graphs.

While all of this knowledge available in the public domain drives growing interest in doing 
research regarding the Semantic Web, there is also a need to have a basic understanding of 
how data is structured (RDF) and how to access this data (SPARQL). These requirements 
represent a barrier-to-entry for non-expert users. Hence these barriers lead to the broad and 
complex challenge of developing intuitive and easy-to-use interfaces for end-users.

Many solutions have emerged to approach this issue, among which natural language interfaces 
such as Question Answering Systems (QAS)~\cite{qa:survey-BOUZIANE2015366, qa:intro-UngerFC14, qa:nn-qakg-Chakraborty19} 
have been receiving much attention. 
These systems aim to answer questions posed by humans in natural language, extracting the 
answer from one or more sources. There are QASs able to retrieve answers from an unstructured 
collection of natural language~\cite{qa:survey-BOUZIANE2015366}. However, we are interested in systems able to 
construct their responses by querying structured data, like relational databases or knowledge 
graphs from the Linked Data Cloud.

More specifically, the task of answering natural language questions using knowledge graphs 
is known as Question Answering over Knowledge Graphs (QAKG)~\cite{qa:nn-qakg-Chakraborty19} or Question 
Answering over Linked Data (QALD)~\cite{qa:intro-UngerFC14, qa:qald-Lopezetal2013}. Unger et al.~\cite{qa:intro-UngerFC14} define the QALD task
as follows: \textit{“translate the users’ information need into a form such that they can be evaluated using 
standard Semantic Web query processing and inference techniques.”} Their work also describes 
the types of questions these systems aim to answer, which often focus on definition questions 
(“Who was Tom Jobim?”) or factoid questions. These last ones that can be divided into 
predicative questions (“Who was the first man in space?”), list questions (“Give me all cities 
in Germany”) or boolean questions (“Was Margaret Thatcher a chemist?”). 

Several Question Answering systems have been developed to address some of the main 
challenges in Question Answering, commonly using pattern matching~\cite{qa:pattern-FaderZE13, qa:pattern-LopezFMS12}, 
grammar-based techniques~\cite{qa:grammar-DamljanovicAC10, qa:grammar-2-Marginean17}, 
or graph exploration~\cite{qa:graph-XuFZ14, qa:graph-2-ZouHWYHZ14} approaches. Although these systems have 
shown good results when dealing with a considerable amount of questions, their performance 
decreases~\cite{qa:challenges-semweb-HoffnerWMULN17} with questions involving more complex graph patterns or with vocabulary 
mismatches caused by the user typing different terms to the ones contained in the knowledge 
graph from which the information is being retrieved (this issue is also known as the lexical gap~\cite{semPar:lexical-gap-HakimovUWC15}). One example 
is the question \textit{“Which US player is the highest scorer in World Cups?”} which could require 
complex operations like aggregation and sorting (count goals, sort and retrieve the maximum 
scorer) and might have vocabulary mismatches (it is not made explicit that we refer to FIFA World 
Cups, or the US might not be registered as an abbreviation of United States).

Aside from works that have tried to mitigate these problems~\cite{semPar:lexical-gap-HakimovUWC15, semPar:complex-queries-GliozzoK12}, some approaches 
that rely on Semantic Parsing have shown positive results due to recent advances in Deep 
Learning applied to Natural Language Processing~\cite{semPar:sempar-as-mt-AndreasVC13}. Semantic Parsing is the process of 
mapping a natural language sentence into a formal representation of its meaning~\cite{semPar:sempar-as-mt-AndreasVC13}. Some 
applications include code generation~\cite{semPar:code-gen-RabinovichSK17, semPar:tranx-code-gen-YinN18}, 
automated reasoning~\cite{semPar:ITPKaliszykUV17} or query construction~\cite{semPar:txt-to-sql-RadevKZZFRS18}. 
In particular, Andreas et al.~\cite{semPar:sempar-as-mt-AndreasVC13} discussed how Semantic Parsing could benefit from using 
Machine Translation methods, whereby natural language is “translated” into a structured 
representation. Following their work, some Neural Machine Translation (NMT) approaches 
have brought about a growing interest in applying deep neural networks to Semantic Parsing 
problems~\cite{nmt:CaiXZYLL18, nmt:DongL16, nmt:ZhongCoRR17}. In NMT, pairs of sequences are given as input to a Deep Neural Network 
model, which is expected to learn the translation model. A natural idea is then to try to apply 
the NMT approach for translating Natural Language (NL) to SPARQL, and some works have begun to explore such techniques~\cite{nmt:CoRRLuz18, nmt:nspm-SoruMMPVEN17, nmt:CoRRSoru18}.

There are multiple challenges relating to converting a NL question to its SPARQL query 
representation (NL-to-SPARQL); some of these challenges are directly inherited from the 
original NL-to-NL translation problem, while others are distinct. First, there isn’t a one-to-one 
mapping for every NL question to a SPARQL query. On one hand, there are multiple ways to 
express a question in NL. For example, the question \textit{“How far away is the Earth from the Sun?”} 
can also be rephrased as \textit{“What is the distance between the Sun and Earth?”}. On the other 
hand, questions can be translated to SPARQL in different ways. For example, a question \textit{“What 
is the largest country in Africa?”} might be translated to a query based on population or area, 
where one such translation must be chosen, and where both give different answers. Moreover, 
one SPARQL query has potentially equivalent queries that will return the same results (over any 
data). In the question about US players, for example, we can first retrieve US players and then 
count their scores, or vice versa; thus there is a need to establish some common conventions 
when designing NL-to-SPARQL systems. 

Another issue is the lack of corpora for training NL-to-SPARQL models when compared to 
the enormous amount of documents in different languages that can be found on the Web for 
training NL-to-NL models (e.g. news, blogs, articles, academic documents, etc.). Generating 
data for NL-to-SPARQL is not an easy task considering the need for a basic SPARQL understanding 
to build such datasets, although there is some work regarding automating parts 
of the process of generating NL-to-SPARQL pairs~\cite{dataset:dbnqa-hartmann-marx-soru-2018, dataset:lcquad-TrivediMDL17}. 
Furthermore, the queries required to answer a question change from dataset to dataset, where the SPARQL queries needed for 
DBpedia are not the same as those for Wikidata.

One of the most recent works in regards to using NMT for SPARQL is that 
the Neural SPARQL Machine (NSpM)~\cite{nmt:nspm-SoruMMPVEN17}, which considers SPARQL as a foreign language. 
The main idea is to train an end-to-end learning model to translate any NL expression 
into a sequence of tokens in the SPARQL grammar that expresses a query equivalent to the 
question over a given dataset. Question Answering systems based on neural networks usually 
aim to generate the whole SPARQL query in an attempt to perform the entire process of 
identifying the relevant entities along with deducing the KB properties, triples, and operators 
that would retrieve the expected answer. 

An analysis of the performance of many NMT models on translating NL to SPARQL 
has been performed by Yin et al.~\cite{nmt:nl-to-sparql-Yin19}, where eight deep neural network models were tested 
over different datasets based on questions over DBpedia. One relevant dataset is the \textit{Large 
Complex Question Answering Dataset} (LC-QuAD)~\cite{dataset:lcquad-TrivediMDL17}, consisting of 5,000 complex questions 
based on 38 hand-made templates. Another dataset is the \textit{DBpedia Neural Question Answering} 
dataset (DBNQA)~\cite{dataset:dbnqa-hartmann-marx-soru-2018} that includes almost 900,000 questions based on templates extracted 
from questions of the LC-QuAD dataset and the 7\textsuperscript{th} version of the \textit{Question Answering over 
Linked Data} dataset (QALD-7)~\cite{dataset:qald7-UsbeckNHKRN17}.

Though the performance of these models shows promising potential for constructing meaningful 
and useful SPARQL queries, they present some key limitations relating to the data 
used to train the models. First, despite the fact that many NMT models report positive results 
when evaluated over simple and large datasets like the DBNQA dataset~\cite{nmt:nl-to-sparql-Yin19}, which 
contains questions with little variation in syntax and phrasing, such regular data do not give an accurate 
understanding of the real performance of these systems. In fact, the performance of such models drops 
dramatically when evaluated over more complex questions like the ones included in the LC-QuAD 
dataset, which might not contain enough questions to learn accurately~\cite{nmt:nl-to-sparql-Yin19}. Second, 
the current models are vocabulary-dependent, which means there is no capability for recognizing 
new entities or properties that were not used in the training data~\cite{nmt:nl-to-sparql-Yin19}. These issues lead 
to the constant need to train the model with new, manually created pairs of NL questions 
and SPARQL queries.

The first issue can be addressed by building a more comprehensive dataset that has enough 
cases for an NMT to learn properly while maintaining an abstraction level that allows us to 
respond accurately to complex and diverse questions. Regardless, creating new datasets does 
not necessarily help with the vocabulary dependency issue, unless we have examples using 
every entity and property in the knowledge graph, which seems infeasible to generate in the 
short-to-medium term. Therefore, an important the goal is to maximize the available training examples, 
where there is a need to find an alternative approach to complement the parsing power 
of NMTs with a system that helps to address vocabulary dependency by extracting the 
information NMTs cannot generalize.

For example, NMTs cannot be expected to extract entities from a question and translate them to their identifiers in the KG. 
Developing labelled examples for each entity does not seem feasible, as mentioned before. 
Plenty of solutions have addressed the entity extraction task over Linked Data. In particular, 
the Information Extraction area, which involves the automatic extraction of implicit 
information from unstructured or semi-structured sources, intersects in many ways with the 
Semantic Web~\cite{infExtr:MartinezHL19}. Some examples are systems that perform Named Entity 
Recognition~\cite{ner:LampleBSKD16}, Sequence Labeling~\cite{seqlab:MaH16, seqlab:contextual-emb-AkbikBV18}, 
or Entity Linking~\cite{EL:dbpedia-spotlight-MendesJGB11, EL:aida-tool-YosefHBSW11, EL:tagme-FerraginaS10, EL:opentapioca-Delpeuch19} 
while leveraging Semantic Web resources and/or techniques. 
In particular, Entity Linking (EL) systems aim to perform the entire process of identifying entity names in a 
text, mapping names to KB resources, and disambiguating them depending on the context 
of a given corpus~\cite{EL:survey-WuHH18}. Considering again the question \textit{“Which US player is the highest scorer 
in World Cups?”}, an EL system could effectively identify the resources associated with the 
country US or the FIFA World Cup. Many EL systems~\cite{EL:dbpedia-spotlight-MendesJGB11, EL:aida-tool-YosefHBSW11, EL:tagme-FerraginaS10, EL:opentapioca-Delpeuch19} 
have achieved positive results when linking KB entities over text and these systems tend to generalize well 
over any new corpus. 

Furthermore, these Information Extraction tools can be complemented along with intermediate 
representation of structured queries. An example can be found on a proposed 
Text-to-SQL system~\cite{semPar:txt-to-sql-RadevKZZFRS18}, where given a question in NL, an intermediate representation of 
an SQL query is generated, consisting of a SQL template with slots to be filled later with 
relevant words identified in the question using Named Entity Recognition tools~\cite{ner:dynet-NeubigDGMAABCCC17}.

We see an opportunity to improve state-of-the-art performance for QASs in the context 
of RDF/SPARQL by exploring the idea of combining the parsing capacity of NMT to get an 
intermediate representation of a SPARQL query, with the entity extraction and disambiguation 
power of Entity Linking systems to identify the relevant entities in questions, decreasing 
the dependency of current QA systems on training examples that cover the full vocabulary of a 
knowledge graph. Additionally, there are many challenges to address – as we have previously mentioned 
– like how to deal with different representations of the same question (e.g. paraphrased questions), 
what canonical representation of SPARQL queries we should adopt, how to evaluate 
that a QAS is fulfilling its purposes, among others.	
	%   Hipotesis, Objetivos, Metodologia
	\section{Hypothesis}
In this work, we propose the following hypothesis: “a combination of Information 
Extraction with Semantic Parsing can develop a Question-Answering system that outperforms 
a system that relies only on Semantic Parsing in the Question Answering over Knowledge 
Graphs task”.

In order to measure performance, this work will use metrics based on two perspectives: 
one focuses on the final answers that are derived from Question Answering over Knowledge 
Graphs (QAKG) benchmarks (i.e., a Question Answering-based evaluation), and the second 
focuses on how close is the generated SPARQL query compared with the expected query (i.e., 
a Machine Translation-based evaluation).

The scope of this work will be limited to answering questions in English, but similar 
techniques should be applicable in any other language assuming the availability of similar 
datasets for that language. In the same way, this hypothesis will be explored in the context of 
questions over Wikidata, so the results might differ for other knowledge graphs. Nevertheless, 
the selected approach should be generalizable to other domains. 

\section{Objectives}
\subsection*{General Objective}
% \lipsum[1-3]
We aim to improve upon state-of-art Question Answering systems 
based on Neural Semantic Parsing models by reducing vocabulary dependency on the 
data used in the learning process of such models.
\subsection*{Specific Objectives}
% \lipsum[1-3]
The specific objective is to build a Question-Answering system over Wikidata in English, 
that relies on Entity Linking and Neural Machine Translation systems, with an intermediate 
system that combines both tools. Our initial claim is that such a system can improve upon 
the state-of-the-art Neural Machine Translation approaches found in the literature.  

\section{Methodology}
Accomplishing the proposed objectives involves the following tasks:
\begin{itemize}
    \item Survey previous work regarding Neural Machine Translation, Entity Linking, and 
    Question Answering approaches that rely on Neural Machine Translation.
    \item Define a benchmark that should include QAKG datasets for training, validation and 
    testing along with metrics to compare all involved systems.
    \item Define a baseline system for Question Answering based on Neural Machine Translation.
    \item Define a pipeline process to convert a natural language question into a SPARQL query 
    by combining Entity Linking techniques with Neural Machine Translation.
    \item Implement a Question-Answering system over Wikidata in English based on the designed pipeline.
    \item Validate the proposed system by comparing it with baseline approaches over the 
    proposed benchmark.
\end{itemize}
	
	% 	Contribuciones
	\section{Contributions}
\label{cap1:intro/contributions}
% \lipsum[1-3]
We present the three main contributions anticipated for this work.
\subsection*{Benchmark on Question Answering over Wikidata}
After a bibliographic revision, we will define a benchmark as a combination of three 
components: a baseline system, a set of metrics to compare with the baseline, and one or 
more datasets with which to conduct experiments.

The baseline consists of one of the Neural Machine Translation systems described by 
Yin et al.~\cite{nmt:nl-to-sparql-Yin19}. From the eight models that were compared in this work, 
the ConvS2S model~\cite{nmt:convS2S-GehringAGYD17} 
significantly outperforms all the other models . Following these results, a baseline QAS is 
implemented using the Fairseq library~\cite{nmt:fairseq-OttEBFGNGA19}, which includes a ConvS2S implementation over 
Pytorch\footnote{\url{https://pytorch.org/}} ready to use for training and translation. The model is trained using the same 
settings described by Yin et al.

The primary dataset used is the \LCQuADtwo{} dataset, which contains around $30,000$ 
questions over Wikidata~\cite{dataset:lcquad2-DubeyBA019}. A quality check is performed over this dataset, where cases 
that contain either invalid questions or invalid \SPARQL{} queries are filtered. The \DBNQA{} 
dataset~\cite{dataset:dbnqa-hartmann-marx-soru-2018} is considered as well, where a mapping process is applied to obtain queries over 
Wikidata. The queries that cannot be mapped are ignored. The Question Answering over 
Linked Data (QALD)~\cite{qa:qald-Lopezetal2013} dataset is also used as part of this benchmark. In particular, the 
150 questions included in \QALDseven{}~\cite{dataset:qald7-UsbeckNHKRN17} that can be answered over Wikidata are considered. 
Besides these datasets, we build a new dataset of 100 questions over Wikidata. Only \LCQuADtwo{} 
and the mapped version of \DBNQA{} are used for training and validation. The other 
datasets are used only for testing purposes. All datasets are arranged to follow a common 
format, which will permit an easy evaluation of every subtask performed for the proposed 
Question Answering system (Entity Linking, Query Template Generation, Slot Filling) along 
with the main task (Question Answering over Knowledge Graphs).

The metrics used for comparing systems are based on the ones used for comparing Neural 
Machine Translation systems and the ones found on the QALD benchmark. The first set of 
metrics includes the BLEU score, Perplexity, and Accuracy by comparing an exact match on 
the \SPARQL{} query. On the other hand, the QALD benchmark uses Precision, Recall, and F1-score 
over the answers obtained when executing the output \SPARQL{} query. Additionally, 
we propose a fine-grained analysis over each case where predicted queries are evaluated with 
respect to the following components: correct entities, correct slots, and correct query 
templates.

\subsection*{Question Answering system over Wikidata}
\label{cap1:intro/contributions/qaWikidata}
The implementation of the new QA system is divided into the construction of three modules: 
an entity retrieval system, a Query Template generator, and a slot filling intermediate 
system. All systems are developed using Python.

We implement various entity retrieval systems using one or more of the existing Entity 
Linking systems that have APIs available. The first variant is to use each EL system individually, 
which includes DBpedia Spotlight~\cite{EL:dbpedia-spotlight-MendesJGB11}, AIDA~\cite{EL:aida-tool-YosefHBSW11}, 
TAGME~\cite{EL:tagme-FerraginaS10}, and OpenTapioca~\cite{EL:opentapioca-Delpeuch19}. 
All of these systems, except for OpenTapioca, only work for DBpedia; therefore an 
extra mapping layer is implemented to map DBpedia entities to Wikidata entities. Two ensemble 
EL approaches are then proposed: one that prioritizes systems with higher Precision 
and the other that implements a voting mechanism. We keep the variant that performs best 
according to the experiments that are explained in the \textit{Experimental results} subsection.

The Query Template generator is built with using the same model used to implement the baseline QAS. 
However, the training data is adapted to generate Query Templates instead of the complete 
query. This is achieved by removing the entities from the output \SPARQL{} queries included 
in the selected datasets such that the entities can rather be found by Entity Linking.

Additionally, a Slot Filling model is trained using a Sequence Labeling model of the Flair 
library~\cite{seqlab:flair-AkbikBBRSV19}. Intuitively speaking, a Query Template may have 
multiple slots and multiple entities, where the Slot Filling model decides which entity should 
fill which slot. This process also requires building training data based on the selected datasets. 
Having finished the three modules, the Question Answering system is implemented by 
connecting these modules.

\subsection*{Experimental results}
\label{cap1:intro/contributions/expResults}
We conduct several experiments for validating each implemented module (Entity Linking, 
Slot Filling, and Query Template Generation) along with experiments over the defined 
benchmark for the end-to-end Question Answering process.

The Entity Linking systems are compared using Precision, Recall, and F1-score on the 
entities for each case on the dataset used for training. Testing is conducted over \QALDseven{} and 
our proposed dataset. The slot filling system is validated using Precision, Recall, and F1-score 
over the identified BIO labels (a common tagging format for sequence labeling tasks) over 
\LCQuADtwo{} and the mapped \DBNQA{} dataset. The query generator system is validated 
using BLEU score, Perplexity, and Accuracy over \LCQuADtwo{} and the mapped \DBNQA{} 
dataset. When training the query generator system, many split methods are tested according 
to the methodology proposed by Finegan-Dollak et al.~\cite{semPar:txt-to-sql-RadevKZZFRS18} used for Text-to-SQL systems. 
The end-to-end Question Answering system is tested over all the datasets using the metrics 
described in the \textit{Benchmark on Question Answering over Wikidata} subsection.	
	%   Estructura del trabajo
	\section{Work Structure}
This worked is divided into the following chapters:

\begin{enumerate}
    \item In Chapter~\ref{cap2:theoFrame}, we describe the theoretical framework 
    enclosed on this work. This chapter the Semantic 
    Web, Information Extraction methods and how they relate with Semantic Web 
    technologies, Semantic Parsing applied to translating natural language to 
    \SPARQL, and the current state and challenges of the Question Answering over 
    Knowledge Graphs task.
    \item In Chapter~\ref{cap3:system}, we give an overview of the proposed Question 
    Answering system for this work. This includes a general explanation of the 
    pipeline proposed to generate a \SPARQL{} query, and more specific details on how 
    each component is designed.    
    \item In Chapter~\ref{cap4:experimentalDesign}, we go into details about the 
    experiments we run in this work. We present the research questions we aimed to answer, 
    the baseline we compare our system with, and the metrics used to quantify the 
    performance of each system.    
    \item In Chapter~\ref{cap5:results}, we present the results derived from running the 
    proposed experiments. Aside from that, we include a brief discussion and analysis of 
    the results.    
    \item In Chapter~\ref{cap6:conclusions}, we summarize the conclusion of this work, 
    discuss its limitations and the future work regarding Question Answering over 
    Knowledge Graphs.
    
\end{enumerate}
	
\end{intro}
% Marco teorico
%   RDF y SPARQL
%   Entity Linking
%   Slot Filling
%   Redes Neuronales
%   Question Answering
\chapter{Theorical Framework}
	\label{cap2:theoFrame}
	% Semantic Web
	\section{Semantic Web}
\lipsum[1-1]
	
	% Information Extraction
	\section{Information Extraction}
\lipsum[1-1]
		
	% Semantic Parsing
	\section{Semantic Parsing}
\label{cap2:theoFrame/semPar}
As mentioned by Kamath and Das~\cite{semPar:KamathD19}, Semantic Parsing is defined as the 
mapping from a natural language utterance into a semantic representation. These 
representations usually refer to logical forms, meaning representations or programs, which 
are executed over an underlying context such as relational tables or Knowledge Graphs. This 
execution yields a desired output like an answer to a question. For example, given a question 
in natural language, a semantic parser can aim to generate a valid \SPARQL{} query based on the 
Wikidata’s ontology grammar that produces the correct answer when executed over a Wikidata 
endpoint.

The first component of a Semantic Parsing framework is the \textbf{language} to represent 
logical forms or meaning representations such as logic based 
formalisms~\cite{semPar:LiangBLFL16,semPar:ArtziFZ13}, graph based 
formalisms~\cite{semPar:BanarescuBCGGHK13,semPar:OepenKMZCFHU15} or programming 
languages~\cite{semPar:FeurerKESBH15}. In particular, we focus on query languages such as \SQL{} 
or \SPARQL{}. Another component is the \textbf{grammar}, which is a set of rules used to decide the 
expressivity of a semantic parser. An example is the Combinatory Categorial Grammar for complex 
structured queries~\cite{semPar:steedman1996}. A last component is the \textbf{underlying context}, 
which is the environment over which the output mappings are interpreted or executed. Knowledge 
Graphs such as Wikidata or DBpedia serve as examples of an underlying context.

The early attempts for Semantic Parsing were systems based on rules or statistical techniques. 
Among the rule-based systems, systems could be based on pattern matching~\cite{semPar:Johnson84a} 
or syntax-based systems~\cite{semPar:Woods73}. Though their implementation is simple, 
rule-based systems tend to be domain specific, thus hard to adapt to other domains. On the 
other hand, statistical models are able to train given examples of input-output pairs from 
any domain. Many approaches require a lexicon as a-priori 
knowledge~\cite{semPar:ZelleM96, semPar:ThompsonM03}, which is used to extract relevant 
semantic or syntactic information. Since these examples are usually manually annotated or 
require complex annotations, statistical models are hard to scale. There is also an issue 
with data sparsity, so these models only work in narrow domains.

Some of the most recent approaches that have emerged are based on Sequence-to-Sequence (Seq2seq) 
models, which usually uses an encoder-decoder framework based on neural networks. Some 
approaches implement an end-to-end paradigm where an intermediate representation is not 
needed to deliver a meaning representation; thus they do not rely on lexicons, templates or 
manually generated features. Though traditional approaches are able to better model and 
leverage the in-built knowledge of logic compositionality, approaches based on sequence 
models outperform traditional approaches due to the fact that Seq2seq-based models generalize 
better with more complex and longer sentences~\cite{semPar:JiaL16}. Furthermore, Seq2seq-based 
models can also generalize across domains~\cite{semPar:KamathD19}.

In the following subsections, we discuss in more depth how systems based on 
Sequence-to-Sequence models work. First, we will briefly explain Sequence-to-Sequence models 
along with the approach we will use in this work: the Convolutional Sequence-to-Sequence model. 
Then, we will introduce Neural Machine Translation systems and how these models can be used 
for the task of translating natural language questions to \SPARQL{} queries.

\subsection{Sequence to Sequence models}
\label{cap2:theoFrame/semPar/seq2seq}
The Sequence-to-Sequence (Seq2seq) model was first introduced by 
Cho et al.~\cite{seqlab:ChoMBB14} for statistical Machine Translation. 
They proposed a neural network model based on an encoder-decoder framework 
which is based on recurrent neural networks 
(RNNs)~\cite{semPar:werbos1990, semPar:rumelhart1986,seqlab:HochreiterS97}. More details 
about RNNs can be found in Appendix~\ref{appendix:neuralNetworks}.

In a Seq2seq architecture, the encoder converts a variable-length sequence into a fixed-length 
vector representation (i.e., it encodes the input sequence into a context vector) which is 
passed through to the decoder that transforms this fixed-length vector representation back 
into a variable-length sequence (i.e. decodes a context vector back to another output 
sequence). Figure~\ref{fig:seq2seqModel} illustrates graphically how a Seq2seq looks, where 
the length $T$ of the input sequence does not necessarily equal the length $T'$ of the output 
sequence.

\begin{figure}[!h]
    \centering
    \includegraphics[scale=.5]{imagenes/2_theorical_framework/semantic_parsing/seq2seqModel.PNG}
    \caption{Sequence to Sequence model~\cite{seqlab:Graves2012-385}.}
    \label{fig:seq2seqModel}
\end{figure}

Technically, the model is learning a conditional distribution over a variable-length sequence 
conditioned on yet another variable-length sequence $p(y_1,\ldots,y_T'|x_1,\ldots,x_T)$. The 
encoder is an RNN that reads each symbol of an input sequence $x$ sequentially. While it is 
reading the current symbol on each step $t$, the hidden state $h_{t}^e$ of the RNN changes are
described as:

\[
    h_{t}^e= f(h_{t-1}^e,x_t)
\]

After reading the end of the sequence, the hidden state of the RNN is the summary $c$ of the 
whole input sequence, also known as its context vector. Then, the decoder is another RNN 
trained to generate the output sequence by predicting the next symbol $y_t$ given the hidden 
state $h_{t}^d$. This prediction is also conditioned on the previous predicted symbol 
$y_{t-1}$ and on the context vector $c$. Then, the hidden state of the decoder is defined 
for the step $t$, where $f$ is usually the \textit{sigmoid} function:

\[
    h_{t}^d= f(h_{t-1}^d,y_{t-1},c)
\]

Similarly, the conditional distribution of the next symbol, where $g$ is commonly a 
\textit{softmax} function since a valid probability must be produced, is defined as follows:

\[
    P(y_t|y_{t-1},y_{t-2},\ldots,y_1,c) = g(h_{t-1}^d,y_{t-1},c)
\]

Both components of the sequence model are jointly trained to maximize the following 
conditional log-likelihood function:

\[
    max_{\theta} \: \frac{1}{N} \sum_{n=1}^N log \; p(y_n|x_n)
\]

Once the model is trained, it can be used to generate a target sequence given an input 
sequence. Though Seq2seq models were originally designed based on RNNs, other variants have 
emerged~\cite{semPar:SutskeverVL14,nmt:DongL16}; among the more modern ones, a recent work 
introduces a sequence learning approach based on convolutional neural networks, which has 
shown to outperform many RNN-based models in the task of \NLtoSPARQL~\cite{nmt:nl-to-sparql-Yin19}.

\subsubsection{Convolutional Sequence to Sequence Model}
\label{cap2:theoFrame/semPar/seq2seq/convS2S}
A Seq2seq model based completely on convolutional neural networks (CNNs) is 
proposed by Gehring et al.~\cite{nmt:convS2S-GehringAGYD17}, called the Convolutional 
Sequence-to-Sequence model (ConvS2S). For this subsection we will assume a basic 
understanding of CNNs, where more details about this topic can be found in 
Appendix~\ref{appendix:neuralNetworks}.

Since CNNs do not receive the input as a sequence like RNNs do, a \textbf{position embedding} 
is proposed. First, the input elements $x=(x_1,\ldots,x_m)$ are embedded in distributional 
space as $w=(w_1,\ldots,w_m)$, where $w_j$ is a column in an embedding matrix $D$. These 
embeddings are combined with an absolute position vector $p=(p_1,\ldots, p_m)$, which 
indicates the position of the word in the sequence, in order to establish a sense of order in 
the input. From this combination the input element representation $e=(w_1+p_1,\ldots, w_m+p_m)$ 
is obtained. The output elements generated by the decoder are built using a similar 
representation.

To compute intermediate states, a simple \textbf{convolutional block structure} is used for both 
encoder and decoder, where such blocks are also referred to as \textit{layers}. These intermediate 
states are based on a fixed number of input elements whose output for the l-th block are 
denoted as $z^l=(z_1^l,\ldots,z_m^l)$ for the encoder network and $h^l=(h_1^l,\ldots,h_m^l)$ 
for the decoder network. Each block contains a one dimensional convolution followed by a 
non-linearity.

Each convolution kernel is parameterized as $(W, b_w)$ and takes as input $X$, which is a 
concatenation of $k$ input elements embedded in $d$ dimensions, and maps them to a single 
output element $Y$. Subsequent blocks operate over the $k$ output elements of the previous 
block. The gated linear unit (GLU)~\cite{semPar:DauphinFAG16} was chosen as the non-linearity 
to apply over the output of the convolution $Y$. This gating mechanism permits to control which 
input values of the current context are relevant. Aside from that, to enable deep convolutional 
networks, residual connections are added from the input of each convolution to the output of the 
block~\cite{semPar:HeZRS15}.

The input of the encoder network is padded to match the output length of the convolutional 
blocks for each block. The padding is done by adding $k - 1$ zero vectors on both the left 
and right side of the input to then remove k elements from the end of the convolution output. 
The same padding cannot be done for the decoder network since no future information is known 
beforehand~\cite{semPar:OordKK16}.

A linear mapping is added for projecting between the embedding size and the convolution 
outputs. This mapping is applied to $w$ when feeding embeddings to the encoder network, to the 
encoder output $z_j^u$, to the final block of the decoder just before the softmax $h^L$, and 
to all decoder blocks $h^l$ before computing the attention score, which is explained later. 

Lastly, a distribution is calculated over the $T$ possible next target elements $y_{i+1}$ by 
transforming the top decoder output $h_i^L$ via a linear layer with weights $W_o$ and bias 
$b_o$, as follows:

\[
    p(y_{i+1}|y_1,\ldots,y_i,x)=softmax(W_o h_i^L + b_o)
\]

Besides the convolutional block structures, a separate \textbf{attention mechanism} is implemented 
for each decoder layer. Attention allows the model to focus on the relevant parts of the 
sentence for each time step. The entire process is illustrated in 
Figure~\ref{fig:convBlockStruct}. The calculation of the attention starts in the bottom left 
part of Figure~\ref{fig:convBlockStruct} with a combination between the current decoder state 
$h_i^l$ and the embedding of the previous target element $g_i$, which is defined as:

\[
    d_i^l= W_d^l h_i^l + b_d^l + g_i
\]

Then looking at the center part of Figure~\ref{fig:convBlockStruct}, for a decoder block $l$, 
the attention $a_{ij}^l$ of state $i$ and source element $j$ is computed as a dot-product 
between the decoder state summary $d_i^l$ and each output $z_j^u$ of the last encoder block $u$:

\[
    a_{ij}^l = \frac{exp(d_i^l \cdot z_j^u)}{\sum_{i=1}^m exp(d_i^l \cdot z_j^u)}
\]

Subsequently, the conditional input $c_i^l$ to the current decoder block is a weighted sum of 
the encoder outputs as well as the input element embeddings $e_j = w_j + p_j$ which corresponds 
to the center right part of Figure~\ref{fig:convBlockStruct}, and is defined as:

\[
    c_i^l = \sum_{j=1}^m a_{ij}^l (z_j^u + e_j)
\]

Finally, the conditional input $c_i^l$ is added to the output of the corresponding decoder layer 
$h_i^l$, as seen in the bottom right part of Figure~\ref{fig:convBlockStruct}. Compared with the 
classical single step attention, this proposal is named a \textit{multi-step attention} mechanism 
since each step takes into account the attention history of the previous time steps, based on how 
conditional inputs are computed. Thus, information does not struggle to survive several steps 
as happens with recurrent networks.

\begin{figure}[!h]
    \centering
    \includegraphics[scale=.5]{imagenes/2_theorical_framework/semantic_parsing/conS2SModel.PNG}
    \caption{Convolutional block structure with a multi-step attention mechanism~\cite{nmt:convS2S-GehringAGYD17}.}
    \label{fig:convBlockStruct}
\end{figure}

Some \textbf{normalization strategies} are applied to stabilize the learning process. First, 
the input and output of a residual block are summed with $\sqrt{0.5}$ to halve the variance of 
the sum. Second, the conditional input $c_i^l$ are scaled by $m\sqrt{1/m}$ to counteract any 
change in variance. Lastly, for convolutional decoders with multiple attention, the gradients 
for the encoder blocks were scaled by the number of attention mechanisms used, excluding source 
word embeddings.

Besides scaling, a \textbf{weight initialization} is also done to keep variance retained. 
Embeddings are initialized from a normal distribution with mean 0 and standard deviation 0.1. 
Weights from layers whose output is not directly fed to a GLU are initialized from 
$\mathcal{N}(0, \sqrt{1/n_l})$, where $n_l$ is the number of input connections to each neuron. 
This helps to maintain a variance of the input with a normalized distribution. Layers followed 
by a GLU are initialized from $\mathcal{N}(0, \sqrt{4/n_l})$, which is a weight initialization 
scheme based on works by He et al.~\cite{semPar:HeZRS15} and Glorot \& Bengio~\cite{semPar:GlorotB10}. 
Biases are uniformly set to zero. Lastly, dropout is applied to the input on some layers with a 
probability of p of being retained and so scaled by $1/p$, or setting them to zero 
otherwise~\cite{semPar:SutskeverVL14}.

\subsection{Natural Language to SPARQL}
\label{cap2:theoFrame/semPar/nlToSparql}
Though most works have focused on translating natural language to \SQL{} queries~\cite{nmt:CaiXZYLL18,nmt:ZhongCoRR17}, 
recent work has also addressed the translation of natural language to \SPARQL{}. In particular, 
Neural Machine Translation (NMT), which involves Machine Translation systems based on Neural 
Networks, has been used to develop systems that translate questions to \SPARQL{} queries. An 
evaluation of eight different NMT models was performed by Yin et al.~\cite{nmt:nl-to-sparql-Yin19}. 
The NMT models included in this work were based on the aforementioned encoder-decoder Seq2seq 
architecture. Among these models, six of them are based on recurrent neural networks (RNN), 
while the remaining two are based on the ConvS2S model~\cite{nmt:convS2S-GehringAGYD17} and 
the Transformer model~\cite{semPar:VaswaniSPUJGKP17} respectively.

The systems based on RNNs include a baseline and many variants of the same LSTM architecture 
and different types of attention mechanisms. As a baseline, a Neural SPARQL Machine 
(NSpM)~\cite{nmt:nspm-SoruMMPVEN17, nmt:CoRRSoru18} is used, which consists of a 2-layer LSTM 
model with no attention mechanisms. Then, two variants of the NSpM are used with two different 
types of attention: global attention and local attention. Global attention uses the entire 
input sentence to calculate an attention vector, which complements the context vector output by 
the encoder~\cite{nlToSparql:BahdanauCB14}. On the other hand, local attention only uses a 
fixed size window around every word of the sequence to calculate a scoped attention vector per 
word~\cite{nlToSparql:LuongPM15}. Another model is based on a proposal from 
Luong et al.~\cite{nlToSparql:LuongPM15}, which consists of a 4-layer LSTM model with local 
attention. Lastly, the Google Machine Translation (GNMT) architecture~\cite{nlToSparql:WuSCLNMKCGMKSJL16} 
is included with two different variants: a 4-layer LSTM and a 8-layer LSTM, both with local 
attention. The only difference between Luong’s model and the GNMT is that the second model 
includes residual connections from the third layer and uses a bidirectional LSTM in the first 
layer of the encoder.

\subsubsection{Neural Machine Translation}
\label{cap2:theoFrame/semPar/nlToSparql/nmt}
Though the architecture of each model varies, the encoding of \SPARQL{} queries used in all 
approaches is the same as that proposed by Soru et al.~\cite{nmt:nspm-SoruMMPVEN17}. Their 
encoding suggests that URIs are abbreviated using their prefixes  followed by an underscore; 
brackets and dots are replaced by their verbal description, and \SPARQL{} keywords are lower-cased. 
For example, a query over Wikidata is shown in Listing~\ref{lst:decodedWikidataSparqlExample}, 
when after an encoding conversion, it is converted to an encoded query as seen in 
Listing~\ref{lst:encodedWikidataSparqlExample}.

\begin{sparqlcode}[%
    caption={\SPARQL{} query before encoding.}, 
    label={lst:decodedWikidataSparqlExample}]
PREFIX wd: <http://wikidata.org/wiki/>
PREFIX wdt: <http://wikidata.org/prop/direct/>

SELECT ?sbj 
WHERE {
    ?sbj wdt:P19 wd:Q201007 .
    ?sbj wdt:P166 wd:Q37922 .
}
\end{sparqlcode}

After training, the system will output an encoded form that can be transformed back to the 
original \SPARQL{} representation. 

\begin{sparqlcode}[%
    caption={\SPARQL{} query after encoding (note it excludes the \texttt{PREFIX} clauses).}, 
    label={lst:encodedWikidataSparqlExample}]
select var_sbj where brack_open var_sbj wdt_p19 wd_q201007 sep_dot var_sbj wdt_p166 wd_q37922 sep_dot brack_close
\end{sparqlcode}

The evaluation metrics typically used to compare these systems are string-matching Accuracy, 
BLEU score and Perplexity. The string-matching Accuracy is used to measure the amount of exact 
matches delivered by each system. The global Accuracy is then the percentage of cases that are 
syntactically equal to the expected answer. This metric is particularly useful when measured 
over a dataset based on \SPARQL{} templates, giving insights into whether or not the expected 
\SPARQL{} form is being correctly generated.

The Bilingual Evaluation Understudy (BLEU) score is used to measure how similar two sentences 
are by using a geometric average of modified n-gram Precision~\cite{nlToSparql:PapineniRWZ02}, 
which is represented by the following formula:

\[
    BLEU = BP \cdot exp(\sum_{n=1}^N w_n \; log \: p_n)    
\]

The modified Precision $p_n$ for each candidate counts the number of times an n-gram occurs in a 
reference translation(s), takes the maximum count of each n-gram among the reference(s), and 
then clips the count of each n-gram in the candidate translation to the maximum count. To avoid 
short candidate translations getting higher scores than desired, a brevity penalty (BP) is 
applied which is set to 1 if the candidate length $c$ is larger than the maximal reference 
length $r$ or set to $exp(1-r/c)$ otherwise. The $w_n$ represents weights for each modified 
Precision. By default $w_n=\frac{1}{N}$ and $N=4$. In this case, the BLEU score goes from 0 to 
100, where a score closer to 100 means the model is performing better. Note that BLEU does not 
account for word order. 

\textbf{Perplexity} is used to understand the model’s intrinsic behavior based on a cross 
entropy H(p,q) which is defined as follows:

\[
    H(p,q)= -\sum_x p(x) \; log \: q(x)
\]

\noindent where $p$ represents the target probability distribution and q is the estimated probability 
distribution. Their similarity is defined by all possible values $x$ in the distribution. In 
this case, $p$ is the one-hot encoding vector of the target vocabulary and q is deduced from 
the result of the output softmax layer. Then the Perplexity is defined as the exponentiation 
of the cross entropy:
\[
    perplexity(p,q)=2^{H(p,q)}
\]

According to the experiments conducted by Ying et al.~\cite{nmt:nl-to-sparql-Yin19}, the model 
that performs best was the ConvS2S model. Although many datasets were used in their experiments, 
we are only interested in \LCQuADone~\cite{dataset:lcquad2-DubeyBA019} and 
\DBNQA~\cite{dataset:dbnqa-hartmann-marx-soru-2018}, two datasets that represent opposite traits 
of a Question Answering Dataset: the \LCQuADone{} dataset contains few questions (5000) but with high 
complexity and high variance of its questions while the \DBNQA{} dataset contains a huge volume of 
questions (nearly $900,000$) but it lacks variety in its questions. We explain with more details 
what we understand by a \dquotes{good dataset} in the \textit{Question Answering} section~\ref{cap2:theoFrame/qakg}. 
The results of the three best models among the eight selected over the mentioned datasets are 
shown in Table~\ref{table:nmtYinResults}.

\begin{table}[h!]
    \centering
    \resizebox{\textwidth}{!}{%
    \begin{tabular}{|l|ll|ll|ll|ll|ll|ll|}
    \hline
    \multirow{2}{*}{} & \multicolumn{4}{c|}{\textbf{Perplexity}}                                                                    & \multicolumn{4}{c|}{\textbf{BLEU Score}}                                                                  & \multicolumn{4}{c|}{\textbf{Accuracy}}                                                                    \\ \cline{2-13} 
                      & \multicolumn{2}{c|}{\textbf{LC-QuAD 1}}              & \multicolumn{2}{c|}{\textbf{DBNQA}}                  & \multicolumn{2}{c|}{\textbf{LC-QuAD 1}}             & \multicolumn{2}{c|}{\textbf{DBNQA}}                 & \multicolumn{2}{c|}{\textbf{LC-QuAD 1}}             & \multicolumn{2}{c|}{\textbf{DBNQA}}                 \\ \hline
    \textbf{Models}   & \multicolumn{1}{l|}{\textbf{Train}} & \textbf{Valid} & \multicolumn{1}{l|}{\textbf{Train}} & \textbf{Valid} & \multicolumn{1}{l|}{\textbf{Valid}} & \textbf{Test} & \multicolumn{1}{l|}{\textbf{Valid}} & \textbf{Test} & \multicolumn{1}{l|}{\textbf{Valid}} & \textbf{Test} & \multicolumn{1}{l|}{\textbf{Valid}} & \textbf{Test} \\ \hline
    LSTM\_Luong       & 1.12                                & 4.92           & 1.90                                 & 2.15           & 52.43                               & 51.06         & 77.64                               & 77.67         & 0                                   & 0             & 34                                  & 34            \\
    ConvS2S           & 1.14                                & 3.25           & 1.12                                & 1.25           & 61.89                               & 59.54         & 96.05                               & 96.07         & 8                                   & 8             & 85                                  & 85            \\
    Transformer       & 1.16                                & 3.15           & 2.21                                & 3.34           & 58.99                               & 57.43         & 68.68                               & 68.82         & 7                                   & 4             & 3                                   & 3             \\ \hline
    \end{tabular}%
    }
    \caption{Performance comparison of three models included in Yin et al.\cite{nmt:nl-to-sparql-Yin19}.}
    \label{table:nmtYinResults}
    \end{table}

Based on the Perplexity results, there is a serious overfit for all models over the \LCQuADone{} 
dataset which Yin et al. attributes to the small size of the dataset and its complex questions. 
No evident overfit is spotted over \DBNQA, and the ConvS2S shows better results which is 
reflected in the other results as well. The BLEU score results reflect that ConS2S again 
outperformed all other models and shows a correlation between Perplexity and BLEU, especially 
when looking at \DBNQA{} results. Finally, Accuracy results clearly show that RNN based models and 
the Transformer do not perform positively in any case when compared to the ConvS2S model. 
However, the results over the \LCQuADone{} dataset shows there is still a challenge regarding 
the handling of more complex questions among all NMT models.

	
	% Question Answering over Knowledge Graphs
	\section{Question Answering over Knowledge Graphs}
\label{cap2:theoFrame/qakg}
Question Answering systems respond to the need to access information on the Web without 
detailed knowledge of Semantic Web technologies such as how data is structured (\RDF{}) or how to 
access data (\SPARQL{}). In particular, Question Answering systems (QASs) provide end-users with an 
intuitive and easy-to-use user interface, which hides the complexity behind the Semantic Web 
standards~\cite{qa:intro-UngerFC14}. These systems differ from traditional search engines such 
as Google in the final objective: search engines only return documents in which the answer can be 
potentially found, whereas a Question Answering system aims to return precise answers~\cite{qa:LopezUSM11}.

When we mention the task of Question Answering over Knowledge Graphs (KGQA), we refer to 
receiving a natural language question and returning an answer retrieved from one or more 
Knowledge Graphs (e.g. the first man to walk on the moon is Neil 
Amstrong\footnote{\url{https://www.wikidata.org/wiki/Q1615} in Wikidata.}). Though there is work 
that includes a wider context along with the question, such as hybrid questions or chain of 
questions, we focus only on the problem of responding to an individual question without further 
context besides the question itself.

Another consideration for defining the scope of Question Answering is the question and answer 
type a system aims to respond to/with. Among the types of questions for which a Question 
Answering system usually provides an answer are:

\begin{itemize}
    \item \textbf{Definition question}, refers to a definition of a subject or object (e.g. 
    \dquotesit{Who was Violeta Parra?}).
    \item \textbf{Factoid questions}, which are related to facts. This type of questions 
    includes three different variants:
    \begin{itemize}
        \item \textbf{Predicative questions} that refer to a specific object related to a predicate 
        such as who, what, where, how (e.g. \dquotesit{Who was the first man in space?}, 
        \dquotesit{What is the highest mountain in Chile?}).
        \item \textbf{List questions} that refer to all the answers that fulfill the fact being 
        asked (e.g. \dquotesit{Give me all the countries in America}).
        \item \textbf{Boolean questions} that refer to whether the fact being questioned is true 
        or not (e.g. \dquotesit{Was Gabriela Mistral a poet?}).
    \end{itemize}
\end{itemize}

In the context of Linked Data, a Question Answering system is limited to the information the 
available knowledge graphs can represent. For instance, questions not based on specific facts 
such as process questions (e.g. \dquotesit{How do I make a lemon pie?}) or opinion questions 
(e.g. \dquotesit{What do most Chileans think of global warming?}) cannot be answered.

Questions can also be classified depending on the answer type expected. One example is the work 
of Li et al.~\cite{qa:LiR02}, which classifies questions given five high-level categories: 
\textbf{entities} (e.g. event, color, animal, plant), \textbf{descriptions} (e.g. definition, 
manner, reason), \textbf{humans} (e.g. individual, group), locations (e.g. city, country, 
mountain) and numbers (e.g. count, date, distance, size). Another way is to classify questions 
according to their \textbf{focus} and \textbf{topic}, representing what the question is about. 
For instance, the question \dquotesit{What is the height of Aconcagua mountain?} focuses on the 
property height in the topic of geography while the question \dquotesit{What is the best 
height-increasing drug?} focuses on the same property, but a different topic, which is medicine.

According to Fu et al.~\cite{qa:FuQTLYS20abs-2007-13069}, current systems do not struggle with 
answering simple questions, i.e. questions that only require one subject-predicate-object 
triple fact, but complex questions are still a challenge for KGQA. A complex question usually 
is referred to as three types: questions under specific conditions (e.g. \dquotesit{Who was the 
president of Chile during the Great Recession of 2008?}), questions with more intentions (e.g. 
\dquotesit{Give the names of the Chilean national football team and the number of goals they have 
scored}), and questions requiring constraint inference (e.g. \dquotesit{What is the most 
expensive movie starring Pedro Pascal?}).

\subsection{KGQA approaches}
\label{cap2:theoFrame/qakg/approaches}
Though there is no standard approach for KGQA, most techniques or methods proposed follow a 
similar four-stage pipeline~\cite{qa:core-techniques-DiefenbachLSM18}. First, a question 
analysis stage includes techniques that extract information from the current question using 
purely syntactic features. Then, the phrase mapping stage defines techniques that identify KG 
resources for each named entity and its dependencies. Next, the disambiguation stage 
encapsulates techniques that rank and determine for each named entity the most relevant 
resources according to the context of the question. Lastly, the query construction stage 
considers techniques used to construct the \SPARQL{} query that should retrieve the final answer. 
In summary, most Question Answering systems are a combination of these four steps and vary with 
respect to which techniques are used to address each phase of the Question Answering process.

On the other hand, KGQA approaches can be classified according to which techniques their models 
are based on. The traditional KGQA models rely on predefined templates and manuals to parse 
questions~\cite{qa:core-techniques-DiefenbachLSM18}. However, these traditional approaches require 
a certain level of knowledge of linguistics, and tend to be difficult to scale. Therefore, recent 
work has focused more on two kinds of approaches: methods based on Information Retrieval methods 
and other ones based on Neural Semantic Parsing.

\subsubsection{Information Retrieval-based methods}
\label{cap2:theoFrame/qakg/approaches/infRetrieval}
Information Retrieval (IR) based models reduce the QA task to binary classification or sorting 
over candidate answers. An IR-based method usually begins by identifying relevant entities 
mentioned in the question. Then, it extracts topic-entity-centric subgraphs where all nodes are 
considered candidate answers. Next, candidates are scored using features that provide semantic 
relevance and help to select the final answers. Depending on how the feature representations 
are built, IR-based methods can be divided into those based on feature engineering and those 
based on representation learning.

Methods based on feature engineering rely on manually defined and extracted features. For 
example, Yao et al.~\cite{qa:YaoD14} extract four types of features based on the question’s 
syntactic information, which includes question words, question focus words, topic words, and 
central verbs. These features are used in a classification model to determine the final answer. 
However, building these features is time-consuming and is not able to capture the entire 
semantic information of questions.

Methods based on representation learning convert questions and candidate answers into vector 
representations and reduce KGQA to a semantic matching computation between representations of 
questions and their candidate answers. Some methods incorporate external knowledge to complement 
the representation information and address the incompleteness that the KG might 
have~\cite{qa:XuRFHZ16,qa:SunDZMSC18,qa:TrouillonWRGB16}. Some other methods incorporate a 
multi-hop reasoning process to handle more complex questions~\cite{qa:SukhbaatarSWF15,qa:MillerFDKBW16,qa:QiuWJZ20}.

Overall, IR-based models get rid of manually defined templates and rules, and can follow an 
end-to-end training. However, these methods lack model interpretability and are not able to 
handle complex questions that require constraint inference.

\subsubsection{Neural Semantic Parsing-based methods}
\label{cap2:theoFrame/qakg/approaches/neuSemParsing}
In this context, methods based on Semantic Parsing aim to convert natural languages into 
executable query languages. Methods based on Neural Semantic Parsing (NSP) rely on Neural 
Networks to enhance the parsing capacity and scalability, instead of relying on predefined 
templates or rules. Thus, these models aim to map unstructured questions to intermediate 
logical forms which are later converted to structured queries.

One approach is to use query graphs, which are graphs that encoded questions, have strong 
representation ability and share topological commonalities with knowledge graphs. One example 
is GraphParser, proposed by Reddy et al.~\cite{qa:ReddyLS14} which frames the Semantic Parsing 
problem as a Graph Matching problem. Derived from GraphParser, the Staged Query Graph 
Generation~\cite{qa:YihCHG15} (STAGG) and the Multiple Constraint Query Graph~\cite{qa:BaoDYZZ16} 
(MultiCG) were proposed. While STAGG proposed a different construction of the query graph based 
on a restricted subset of lambda-calculus in the graph representation, MultiCG extends STAGG to 
include more constraint types and operators in order to cover more complex questions. Since all 
these proposals rely on Entity Linking tools to initialize the query graph construction, 
Yu et al.~\cite{qa:YuYHSXZ17} assumes a poor performance of such tools and proposes a 
Hierarchical Residual BiLSTM with the goal of improving entity recognition Accuracy.

On the other hand, other work proposes to use encoder-decoder models to reduce the Semantic 
Parsing task into a Sequence-to-Sequence problem. One example is the Seq-to-Tree model proposed 
by Dong et al.~\cite{nmt:DongL16} that uses a hierarchical tree-structured decoder to capture 
the structure of logical forms. Then, in order to take more advantage of the syntactic 
information of the input question, the Graph-to-Seq model was proposed by Xu et al.~\cite{qa:XuWWYCS18} 
to encode the question as a syntactic graph. A last example is the Neural Symbolic Machine 
proposed by Liang et al.~\cite{qa:LiangBLFL17}, which proposed a lighter supervision approach 
by training a Seq2seq model with reinforcement learning. Nevertheless, none of the examples 
mentioned above have been evaluated in the context of KGQA.

In general, the results of NSP-based models tend to be slightly better than the results from 
IR-based models. Nevertheless, it is still challenging to train a Neural Semantic Parser due to 
the lack of training data.

\subsection{Main Challenges}
\label{cap2:theoFrame/qakg/challenges}
The KGQA task is still an open problem, where many challenges have partially been addressed. 
Some of these challenges are related to the question being asked and others to the answers that 
can be returned. 

There are different ways to express the same question, as there are different ways to represent 
information within Knowledge Graphs. The \textbf{lexical} gap is the difference between the 
vocabulary used in a question and how information is expressed in the Knowledge 
Graph~\cite{semPar:lexical-gap-HakimovUWC15}. The bigger this gap is, the more difficult it is 
for QA systems to locate the correct resources for each named entity identified. 

Some work has tried to address the lexical gap, where most techniques are based on string 
normalization, pattern matching or entailment. For example, string normalization methods, such 
as converting to lower case or stemming (e.g. convert \dquotesit{writing}, to its base form 
\dquotesit{write}) help to reduce the search space when mapping entities. Another example are 
pattern libraries, such as PATTY~\cite{qa:NakasholeWS12} or BOA~\cite{qa:GerberN12}, that 
support pattern matching from a phrase to a resource. Nevertheless, most techniques proposed 
are based on manually defined rules, which is hard to scale or to transfer to other Knowledge 
Graphs.

Another challenge related with input expressivity is the \textbf{ambiguity} bound to each 
question, in other words, questions with different semantic meaning but with similar syntactic 
structure. There are two main types of ambiguities: homonymy, when different concepts are 
represented by a string with the same spelling (e.g. the word \dquotesit{right} can refer to 
\dquotesit{correct} or \dquotesit{direction opposite to left}), and polysemy, when one string can 
represent different but related concepts (e.g. \dquotesit{newspaper} can refer to a 
\dquotesit{printed publication} or to a \dquotesit{media company}). 

Among the approaches used to address ambiguity are corpus-based methods or resource-based 
methods. Usually the corpus-based methods use statistical models from unstructured text 
corpora~\cite{qa:shirai1997,qa:ShenYYJLC11}. On the other hand, resource-based methods take 
advantage of the \RDF{} properties of candidate resources. A score is calculated for each entity 
following the assumption that a better score applies a higher probability of a resource to be 
chosen. Some examples are RVT, which uses Hidden Markov Models~\cite{qa:GiannoneBB13}; CASIA, 
which relies on Markov Logic Networks~\cite{qa:shizhu2014casia}; or Treo~\cite{qa:freitas2011treo,
qa:FreitasOOSC13}, which uses Wikipedia-based semantic relatedness.

Unlike the lexical gap, which affects the Recall of a QA system, ambiguity negatively affects 
its Precision. While some methods aim to reduce the lexical gap, these same methods can make it 
difficult to address ambiguity. It is customary that in the disambiguation stage the systems 
try to balance the effects of both issues.

From challenges that are related to the query construction, the needs of \textbf{complex operators} 
affect the capability of a QA system to respond to more complex questions. The difficulty to 
handle a question increases when several facts have to be identified. Some examples that have 
tried to addresses these issue are YAGO-QA~\cite{qa:AdolphsTSUW11}, which tries to address 
nested questions; PYTHIA~\cite{qa:UngerC11}, which can answer question involving quantifiers, 
comparatives, superlatives and others; and IBM Watson~\cite{qa:GliozzoK12}, which can respond 
to indirect questions and multiple sentences.

As mentioned before, there are certain types of questions that current QA systems struggle to 
handle. One example is that of \textbf{procedural questions} (i.e. describe a procedure), which 
no QA system has been able to solve. Another example relates to \textbf{temporal questions} 
which refer to questions that require inferring temporal relations between events, though some 
works have tried to address temporal questions~\cite{qa:Allen83,qa:FerrandezSKDFNITONMG11,
qa:MeloRN11}. A last example relates to \textbf{spatial questions} that include questions 
referring to locations. The capability of answering spatial questions depends on how the schema 
of the knowledge represents locations (e.g. latitude and longitude), and how QA systems can use 
that information to enrich semantic data~\cite{qa:YounisJTA12,qa:graph-2-ZouHWYHZ14}.

The use of \textbf{templates} is thus more common to construct \SPARQL{} queries that include 
complex operators such as aggregation or filter functions. These \SPARQL{} Query Templates can be 
either manually or automatically created. One template-driven approach is Casia~\cite{qa:shizhu2014casia} 
that uses the question type, named entities and POS tagging techniques to generate graph 
pattern templates from which a \SPARQL{} query is built. Other approaches use manually created 
templates combined with machine learning methods~\cite{qa:AbachaZ12}. 

\subsection{Benchmark \& Datasets}
\label{cap2:theoFrame/qakg/benchmarkDatasets}
In order to measure a Question Answering system performance, the most common parameters used 
across all benchmarks are \textbf{Precision}, \textbf{Recall} and \textbf{F1-score}. These 
three metrics usually are based on a gold standard set per question, which are the entities or 
values expected to be returned by the QA system. Given a question $q$, the formulas for Recall, 
Precision and F1-score are represented in the following formulas:

\begin{equation}
    \begin{aligned}
    Recall(q) &= \frac{\mbox{number of correct system answers for q}}{\mbox{number of gold standard answers for q}} \\
    Precision(q) &= \frac{\mbox{number of correct system answers for q}}{\mbox{number of system answers for q}} \\
    F1(q) &= \frac{2 \ast Precision(q) \cdot Recall(q)}{Precision(q)+Recall(q)}
    \end{aligned}
\end{equation}

As an example, given the question \dquotesit{Which are the primary colors?}, the gold standard 
answer would be \{\texttt{red}, \texttt{green}, \texttt{blue}\}. If a system answer is 
\{\texttt{brown}, \texttt{green}\}, the Precision would be 0.5 since \texttt{green} is the 
only color part of the correct answers among the two colors returned, while the Recall would be 
0.33 because the only color correctly returned among the golden standard set is \texttt{green}.

Similar to Entity Linking system evaluations, the global Precision or global Recall can be 
reported in two ways: as a micro average or a macro average. The F1-score is also used to 
combine Precision and Recall into one measure.

In recent years, many datasets have been developed to include complex questions. In particular, 
we will briefly describe the datasets used in this work whose questions often require complex 
query construction and provide the expected query and result.

\subsubsection{QALD}
\label{cap2:theoFrame/qakg/benchmarkDatasets/qald}
The series of QALD datasets are part of the Question Answering over Linked Data (QALD) challenges, 
which aim to provide an up-to-date benchmark for measuring performance and comparing Question 
Answering systems~\cite{qa:qald-Lopezetal2013}. The QALD challenge is divided into multiple 
tasks that invite QA systems to address different challenges of Question Answering: 
multilinguality, hybrid questions, large-scale question answering and adaptability to other 
data sources. Though most versions of QALD contain questions that can be only answered over 
DBpedia, one of the versions of QALD did contain questions over Wikidata for the task of 
adaptability.

That being said, the $7^{th}$ version of QALD (\QALDseven)~\cite{dataset:qald7-UsbeckNHKRN17} includes a 
dataset of 150 questions over Wikidata, divided into 100 questions for training and 50 for 
testing. The questions comes from a real-world question and query logs, where each question is 
manually annotated and includes a manually specified \SPARQL{} query. Regarding the complexity of 
the questions, about 38\% are considered complex, including questions with counts, superlatives, 
comparatives, and temporal aggregations.

Despite the good quality of the questions of \QALDseven{} and its proximity to real-world questions, 
its limited size means that it is insufficient for training NSP-based models, which require a 
considerable amount of annotations to undergo a successful learning process

\subsubsection{LC-QuAD 1}
\label{cap2:theoFrame/qakg/benchmarkDatasets/lcquad}
The Large-Scale Complex Question Answering Dataset (\LCQuADone{}) is a dataset based on 
DBpedia~\cite{dataset:lcquad-TrivediMDL17}, where 82\% of its questions are considered complex. 
Differently from QALD, the \SPARQL{} queries are built using a relatively small number of 
predefined templates, thus generating a large high-quality dataset with low domain-expert 
intervention. This results in a dataset that contains a total of 5000 questions with their 
corresponding queries.

Their dataset construction pipeline follows a different proposal to the standard one, where 
instead of writing the natural language question and then its logical form (aka the \SPARQL{} 
query), the process is inverted. Thus the process starts by filling 38 hand-made \SPARQL{} 
templates with seed entities and relations from a preferred predicate list to generate specific 
\SPARQL{} queries. After that, a natural language question (NLQ) is deduced, for each generated 
\SPARQL{} query, following a three-step process: a generation of a normalized NLQ by filling a 
Normalized Natural Question Template (NNQT) with the entity and relations names used to 
generate the current query, a verbalization of the normalized NLQ using crowdsourcing tools, 
and a final validation done by an independent reviewer.

The proposal of \LCQuADone{} was a first step to allow NSP-based models to be applied to the field 
of Question Answering in the \RDF{}/\SPARQL{} setting, though most works~\cite{qa:FuQTLYS20abs-2007-13069} 
show that a larger number of examples are still required to train models properly.

\subsubsection{DBNQA}
\label{cap2:theoFrame/qakg/benchmarkDatasets/dbnqa}
The DBpedia Neural Question Answering (DBNQA)~\cite{dataset:dbnqa-hartmann-marx-soru-2018} dataset 
is thus far the largest dataset for Neural Question Answering over DBpedia, with 894,499 annotated 
pairs. The process to construct this dataset is similar to the one proposed for \LCQuADone{}, though its 
templates are extracted from the multilingual \QALDseven{} train dataset that includes 215 questions 
(not the same dataset as the Wikidata \QALDseven{} dataset mentioned before) and the same \LCQuADone 
dataset. This extraction process provides 5217 \SPARQL{} templates. Another difference is the 
approach to generate natural language questions. This dataset originated from the purposes of 
training Neural SPARQL Machines, mentioned in the \textit{Semantic Parsing} section~\ref{cap2:theoFrame/semPar}.

The template extraction process for each case consisted of replacing the concrete entities 
with placeholders. This is the case for the original question, where resource labels are 
replaced, and the query, where entities are replaced. This process functioned differently for 
the two datasets used. For the \QALDseven{} train~\cite{dataset:qald7-UsbeckNHKRN17} dataset the 
resources were manually replaced, while for \LCQuADone{} an automatic script was implemented taking 
advantage of the fact that the resources were marked in the questions. As an example, given the 
question \dquotesit{What are the artists that are born in Stockholm?} which generates the query 
in Listing~\ref{lst:dbnqaSparqlExample}, the template extraction process would return a 
question template \dquotesit{What are the artists that were born in <A>?}, replacing the 
resource Stockholm, and would generate the Query Template shown in Listing~\ref{lst:dbnqaTemplateExample}.

\begin{sparqlcode}[%
    caption={\SPARQL{} query for the question: \dquotesit{What are the artists that are born in Stockholm?}.}, 
    label={lst:dbnqaSparqlExample}]
SELECT DISTINCT ?sbj WHERE {
    ?sbj dbp:placeOfBirth dbr:Stockholm .
    ?sbj rdf:type dbo:MusicalArtist .
}
\end{sparqlcode}

\begin{sparqlcode}[%
    caption={Query Template for the question: \dquotesit{What are the artists that are born in <A>?}.}, 
    label={lst:dbnqaTemplateExample}]
SELECT DISTINCT ?sbj WHERE {
    ?sbj dbp:placeOfBirth <A> .
    ?sbj rdf:type dbo:MusicalArtist .
}
\end{sparqlcode}

Then, the dataset is constructed following a similar approach to the one used for \LCQuADone{}. 
First, a set of entities is chosen per Query Template, where the entities selected are the ones 
that contain the relations contained in the query (e.g. if the property location is contained 
in the query, places would be selected accordingly). Then the selected entities are used to 
generate each \SPARQL{} query with its corresponding NLQ. 

Since the construction of each case does not include a paraphrasing stage, there is almost no 
variation of questions that comes from the same template, i.e. there was no syntactic 
difference for questions with the same purpose. This lack of variation negatively affects the 
generalisation capabilities of NSP-based models trained on this dataset~\cite{qa:BerantL14}.

\subsubsection{LC-QuAD 2}
\label{cap2:theoFrame/qakg/benchmarkDatasets/lcquad2}
Taking into account all the advantages and disadvantages of the aforementioned datasets, a 
second version of the Large-Scale Complex Question Answering Dataset (\LCQuADtwo{}) was 
proposed~\cite{dataset:lcquad2-DubeyBA019}. In particular, this dataset provides around 30,000 
questions over Wikidata which contains complex questions and high diversity among its questions. 
The dataset generation workflow combines semi-automatic question generation along with a 
crowdsourced-based paraphrasing phase.

\SPARQL{} queries are generated given a set of entities, predicates and templates, and do not differ 
much compared to the first version of \texttt{LC-QuAD}. Nevertheless, the \SPARQL{} templates used 
are different from the ones used before. Now the templates are based on one of the 10 types of 
question \LCQuADtwo{} aims to address:

\begin{enumerate}
    \item \textbf{Single fact}: use a single (S-P-O) triple query, e.g. \dquotesit{Who is the 
    screenwriter of Mr. Bean?}.
    \item \textbf{Single fact with type}: the fact is focused on the type of constraint, e.g. 
    \dquotesit{Billie Jean was on the tracklist of which studio album?}.
    \item \textbf{Muti-fact}: use two connected facts, e.g. \dquotesit{What is the name of the 
    sister city tied to Kansas City, which is located in the county of Seville Province?}.
    \item \textbf{Fact with qualifiers}: includes more informative facts stored in qualifier 
    properties, e.g. \dquotesit{What is the venue of Barack Obama’s marriage ?}.
    \item \textbf{Two intentions}: consider questions with two intentions and also rely on 
    qualifiers, e.g. \dquotesit{When and where did Barack Obama get married to Michelle Obama?}.
    \item \textbf{Boolean}: ask whether a fact is true or false, including questions with 
    number as an object, e.g. \dquotesit{Did Breaking Bad have 5 seasons?}.
    \item \textbf{Count}: uses the COUNT keyword to perform a numeric count over a certain fact, 
    e.g. \dquotesit{What is the number of Siblings of Edward III of England ?}.
    \item \textbf{Ranking}: requires counting and sorting in order to rank entities regarding a 
    certain property, e.g. \dquotesit{What is the binary star which has the highest color index?}.
    \item \textbf{String Operation}: applies string operations at a work or character level, 
    e.g. \dquotesit{Give me all the Rock bands that start with the letter R?}.
    \item \textbf{Temporal aspect}: covers temporal properties where various time facts are 
    included in qualifier properties, e.g. \dquotesit{With whom did Barack Obama get married in 1992?}.
\end{enumerate}

Considering that some types of questions might have more than one \SPARQL{} template variation, a 
total of 22 templates are used. 

\begin{figure}[!h]
    \centering
    \includegraphics[scale=.55]{imagenes/2_theorical_framework/question_answering/lcquad2Workflow.PNG}
    \caption{Example of \LCQuADtwo{} question workflow generation~\cite{dataset:lcquad2-DubeyBA019}.}
    \label{fig:lcquad2Pipeline}
\end{figure}

Then, a set of relevant entities and a predicate list is selected based on each \SPARQL{} template, 
meaning that each question type includes different entities and properties (e.g. the property 
\texttt{birthPlace} might not be used for questions that require counting). For each case, a 
subgraph is built based on three factors: an entity, the \SPARQL{} template, and one or more 
suitable predicates. From each subgraph a \SPARQL{} query is generated. Next, each \SPARQL{} query is 
transformed to a normalized natural language question, called a Question Template (QT), which is 
then verbalized and paraphrased in a three-step pipeline based on human turkers from the Amazon 
Mechanical Turk tool (a crowdsourcing tool to perform simple tasks on a large scale). 

The first step is to convert each QT into Verbalized Question (QV), where most of the 
grammatical errors and semantic inconsistencies of the QT are fixed. The second step is to 
paraphrase each QV, where the resulting Paraphrased Question (QP) should preserve the overall 
semantic meaning while changing its syntactic content and structure. An example of the entire 
generation of one case is illustrated in Figure~\ref{fig:lcquad2Pipeline}. The last step is a 
human verification where a comparison is done between each QT with its corresponding QP in 
order to measure the quality of the final result in terms of semantic similarity.

After the entire generation process, a dataset is provided with around 52\% complex questions 
and significant variation in their  question structure. There is, however, a small percentage 
of error derived from using a crowdsourcing tool~\cite{dataset:lcquad2-DubeyBA019}, with questions 
losing part of their semantic intent or questions that are poorly verbalized/paraphrased, which 
should be taken into consideration when using this dataset to train NSP-based models. 
	
% Trabajo relacionado
% Baseline
% Sistema propuesto
%\input{cap2.tex}
%\input{conclu.tex}

% \input{glosario.tex} % opcional

%\appendix
\begin{appendices}
\chapter{Neural Networks overview}
\label{appendix:neuralNetworks}
We provide an overview of the main concepts and techniques relating to Neural Networks as are 
important for this work. Section~\ref{appendix:fundamentalsNN} and~\ref{appendix:recurrentNN} 
are based on Graves~\cite{seqlab:Graves2012-385} work description for Supervised Sequence 
Labeling with Recurrent Networks and Section~\ref{appendix:convolutionalNN} is based 
on~\cite{appendix:OSheaN15}.

\section{Neural Networks Fundamentals}
\label{appendix:fundamentalsNN}
Artificial Neural Networks (ANNs), or simply Neural Networks (NNs), are mathematical models 
inspired by how biological brains process information. Their basic structure is a network of 
small processing units, also seen as nodes, joined to each other by weighted connections. 
Similar to how synapses work in biological neurons, the network is activated by providing an 
input to some or all nodes, which spreads this activation throughout the rest of the network 
along the weighted connections.

\subsection{Multilayer Perceptrons}
Among the different varieties of neural networks, \textbf{Feedforward Neural Networks} are 
structured in an acyclic way, meaning their connections do not form a cycle. One example are 
the multilayer perceptrons (MLP), which are arranged in layers with connections feeding 
forward from one layer to the next. An illustration can be seen in Figure~\ref{fig:multilayerPerceptron}, 
where input data are passed to an \textbf{input layer} and then propagated through one or more 
\textbf{hidden layers} to the final \textbf{output layer}. This kind of architecture is more 
suitable for classification or function approximation tasks~\cite{seqlab:Graves2012-385}.

\begin{figure}[!h]
    \centering
    \includegraphics[scale=.45]{imagenes/insertImage.png}
    \caption{Multilayer perceptron [ref].}
    \label{fig:multilayerPerceptron}
\end{figure}

The process to pass data through layers is known as the \textbf{forward pass} of the network. 
Given an input vector of length $I$, an input layer of the same length would receive this 
input vector where each unit in the input layer calculates a weighted sum. For a hidden unit 
$h$, we refer to this sum as the network input to unit $h$, denote as $a_h$. Denoting $w_{ij}$ 
as the weight from unit $i$ to unit $j$, the formula of $a_h$ for a layer $H_{l}$ is calculated 
using the formula in Equation~\ref{eq:hiddenState}.

\begin{equation} \label{eq:hiddenState}
    a_h = \sum_{h' \in H_{l-1}}^I w_{h'h} \; b_{h'}
\end{equation}

where $b_{h'}$ corresponds to the final activation function of the previous layer. The $b_h$ 
value is calculated by applying an activation function $\theta_h$ over $a_h$, as seen in 
Equation~\ref{eq:finalActivation}. Note that for the first hidden layer, the previous layer 
is the input layer.

\begin{equation} \label{eq:finalActivation}
    b_h = \theta_h(a_h)
\end{equation}

This activation function $h$ can vary, though some of the most common functions used are the 
hyperbolic tangent~\ref{eq:tanh} or sigmoid~\ref{eq:sigmoid} functions. Two important features 
about activation functions: they are non-linear and differentiable. Non-linearity allows the 
network to build more complex internal features (e.g. build more flexible boundaries in a 
classification task), and differentiability is required to perform the backward pass that 
will be mentioned below.

\begin{equation} \label{eq:tanh}
    tanh(x) = \frac{e^{2x}-1}{e^{2x}+1}
\end{equation}
\begin{equation} \label{eq:sigmoid}
    sigmoid(x) = \frac{1}{1+e^{-x}}
\end{equation}

This process of summation and activation is repeated for $L$ hidden layers until reaching the 
output layer, where the output vector $y$ is determined by using the activation of the last 
hidden layer $H_L$. Then, the network input $a_k$ to each output unit $k$ is calculated by 
summing over the units of the connected to it, the same way as is expressed in Equation~\ref{eq:hiddenState}. 
The output activation function to be used depends on the task the network aims to fulfill. 
For simple binary classification, the sigmoid function~\ref{eq:sigmoid} is applied since its 
values between 0 or 1 can be seen as a binary probability $p(z|x)$, with $z$ being the target 
vector. Furthermore, if the classification task includes more than 2 classes, there is a 
convention to have $K$ output units, and normalize the output activations with the softmax 
function~\ref{eq:softmax}. Therefore, the class probability for $C_k$ given the output $x$ is 
represented by Equation 5

\begin{align} \label{eq:softmax}
    p(C_k|x) & = y_k \nonumber \\
        & = softmax(a_k) \nonumber \\ 
        & = \frac{e^{a_k}}{\sum_{k'=1}^K e^{a_{k'}}}
\end{align}

Lastly, a 1-to-K scheme is used to represent the target class $z$ where $z$ is represented as 
a one-hot vector (e.g. if $K=4$, the class $C2$ is represented as $[0, \; 1, \; 0, \; 0]$). 
More formally, the way to express the target probabilities are as follows:

\[
    p(z|x) = \prod_{k=1}^K y_k^{z_k}
\]

In the context of pattern classification, the class label that should be chosen corresponds to 
the most active output unit, i.e. the higher value from all target probabilities.

\subsection{Network Training}
In order to have an idea whether a neural network is working as expected, a loss function is 
used. As per activation functions, the loss function to be used depends on the task the Neural 
Network is performing.  For example, for multiple classification the maximum-likelihood function 
is commonly used as a loss function~\cite{appendix:bishop1995neural}:

\[
    L(y_k, z) = \sum_{k=1}^K z_k ln \; y_k
\]

Neural networks are able to learn, i.e. they can generalize to unseen data, so they can be 
trained by minimizing the loss function L. One of the simplest algorithms to perform such 
training process is the \textit{gradient descent} algorithm. \text{Gradient descent} consists 
of repeatedly taking a small, fixed-size step in the direction of the negative error gradient 
of the loss function, which can also be seen as going in the opposite direction of the 
negative slope of the loss function. Note that we perform gradient descent over a training 
dataset, while we save a test dataset that is not used to train but to evaluate the overall 
performance after training.

Thus, a weight update $\Delta w^n$, also known as gradient , is used to update the the weight 
vector $w^n$ from the $n^{th}$ network layer. The gradient $\Delta w^n$ consists of the 
partial derivative of the loss function when the weight vector $w^n$ varies. This derivative 
is adjusted by a \textbf{learning rate} $\alpha \in [0, 1]$ which limits how quick the training process 
is converging. Then, on each gradient descent iteration $i$ the weight vector $w^n$ is updated as 
follows:

\[
    w_{i}^n = w_{i-1}^n - \Delta w_{i-1}^n = w_{i-1}^n - \alpha \frac{\partial L}{\partial w_{i-1}^n}
\]

The backpropagation technique is commonly used to calculate these 
gradients~\cite{appendix:rumelhart1985learning, appendix:williams1995gradient, appendix:werbos1988generalization}, 
often referred to as the \textbf{backward pass} of the network. Backpropagation consists of 
the repeated application of chain rule for partial derivatives. For example, for a multiclass 
network, the application of the chain rule over the loss function defined as 
$\frac{\partial L(x, z)}{\partial w_{ij}} = \frac{\partial L(x, z)}{\partial a_j} \frac{\partial a_j}{w_{ij}}$, 
which then can be deduced by applying the chain rule again over the unknown gradients. 
Note that this process has to be performed over every weight of each layer on the network.

The training algorithm is repeated until a \textbf{stopping criteria} is met (e.g. stop after 
a fixed amount of steps, when reaching a certain loss threshold, or when failing to reduce the 
loss on a given number of consecutive steps). Usually, this process involves using the entire 
training data more than one time, where an entire pass over the data is known as one \textbf{epoch}. 
By the end of the training process, we expect to have the neural network’s weights such that 
the loss function has reached a value as close as possible to the global minimum when 
evaluating over test examples (i.e. it can predict as best as possible over the training data).

The training process involves many issues that can lead to bad performance, a long time to 
train models, or divergence problems (i.e. training process never ends). One common issue is 
when the gradient descent process gets stuck in local minimums, which can be addressed by 
adding a momentum term to reduce learning inertia~\cite{appendix:plaut1986experiments}. To 
boost training time, many variants of gradient descent have been proposed such as stochastic 
gradient descent or mini-batch gradient descent~\cite{appendix:lecun1998efficient}.

Another issue related with bad performance is \textbf{overfitting}, which causes the network 
not to be able to generalise properly since it \textit{“memorizes”} the data from the training 
dataset. One way to see if a model is overfitted is to check the loss function values 
evolution over the training process for the training and the test set: if the loss for the 
test is not decreasing but instead increasing while the loss for the training set is 
constantly decreasing, the model is getting overfitted.

\begin{figure}[!h]
    \centering
    \includegraphics[scale=.45]{imagenes/appendices/appA_overfitting.PNG}
    \caption{Example of early stopping analysis using validation data~\cite{seqlab:Graves2012-385}.}
    \label{fig:earlyStopppingExample}
\end{figure}

One solution that helps to address the overfitting issue is to use a small portion of the 
training set as a validation set to include an \textbf{early stopping criteria}. This 
validation set is not used to train the network but to perform validation steps every certain 
amount of training steps. Then, the loss values for the validation steps are used to decide 
when to stop training. For example, Figure~\ref{fig:earlyStopppingExample} shows the losses 
over all three data sets (train, validation, test) through the training process Then, we can 
detect that the best weight values are found just before the validation loss stops its 
decreasing tendency and start increasing again (where the “best” stripped vertical line is 
placed). There are other techniques to reduce overfitting known as regularizers based on 
input noise~\cite{appendix:an1996effects, appendix:koistinen1991kernel, appendix:bishop1995neural} 
or weight noise~\cite{appendix:murray1994enhanced,appendix:jim1996analysis}.

Though most of the performance of ANNs relies on learnable parameters (such as layers’ weights), 
there are other parameters that can be set manually to improve the model’s performance, also 
known as \textbf{hyperparameters}. Some hyperparameters could be the number of hidden layers, 
the number of hidden units per layer, the learning rate, among others.

Lastly, it is also important to understand how the network input is represented. When we 
mention the \textbf{input representation}, we refer to the representation of the information 
required to predict the outputs, such as the input vector or the network weights. One 
procedure is input standardisation, which consists of normalizing the components of the input 
vectors to have mean 0 and standard deviation 1 over the training set. This standardization 
does not alter the information but helps to improve performance by limiting the values of the 
input vector to a more suitable range for a standard activation function~\cite{appendix:lecun1998efficient}. 
Note that the validation set and test set have to be standardised using the same distribution 
used for the training set. 

Another procedure is \textbf{weight initialisation}, which helps gradient descent to 
\textit{“break symmetry”} between units~\cite{appendix:lecun1998efficient} and avoid training 
divergence. Weight initialisation then is to initialise weights with either a random 
distribution in the range of small values or a Gaussian distribution with mean 0 and 
standard deviation 0.1.

After reviewing the fundamental concepts needed to understand how neural networks are 
structured and trained, we will review two neural networks architectures used in this work: 
Recurrent Neural Networks and Convolutional Neural Networks.

\section{Recurrent Neural Networks}
\label{appendix:recurrentNN}
A Recurrent Neural Network (RNN) is a generalization of traditional feedforward Neural 
Networks that allows cyclical connections~\cite{seqlab:Graves2012-385}. While MLPs can only 
map from input to output vectors, RNNs can map the entire history of previous inputs to an 
output. Hence, RNN connections allow the network to have “memory” of previous inputs, thus 
influencing the network output. Furthermore, RNNs are more fit for tasks involving sequence 
data, such as text or audio. An example of a RNN architecture can be seen in Figure 1.

\begin{figure}[!h]
    \centering
    \includegraphics[scale=.45]{imagenes/insertImage.png}
    \caption{Recurrent Neural Network architecture example.}
    \label{fig:recurrentNN}
\end{figure}

A standard RNN computes the \textbf{forward pass} the same way as an MLP with a single hidden 
layer, with the difference that the activations arrive at the hidden layer from both the 
current external input and the hidden layer activations from the previous time step. Given a 
sequence of inputs $(x_1,\ldots,x_T)$, with $T$ the input length, the forward pass for an RNN 
with $I$ input units and $H$ hidden units is computed using the following formula:

\begin{equation} \label{eq:hiddenStateRNN}
    a_h^t = \sum_{i=1}^I w_{ih} \; x_i^t + \sum_{h'=1}^H w_{h'h} \; b_{h'}^{t-1}	
\end{equation}

where $x_i^t$ denotes the value of input $i$, $a_j^t$ is the network input to unit $j$, and 
$b_j^t$ is the activation unit of unit $j$, all three at time t. Then, the non-linearity $h$ 
is applied the same way as for an MLP, where functions such as the sigmoid function or 
softmax function are again a common choice:

\begin{equation} \label{eq:activationRNN}
    b_h^t = \theta_h(a_h^t)
\end{equation}

The entire learning process for RNNs is then summarized as a recursive application of the 
Equations~\ref{eq:hiddenStateRNN} and~\ref{eq:activationRNN}, starting at $t=1$. Since 
initial values $b_i^0$ are needed, they can be initialized using the same weight 
initialization methods mentioned above. The output units $a_k$ can be calculated at the same way 
as the hidden activations:

\begin{equation} \label{eq:hiddenStateRNN}
    a_k^t = \sum_{h=1}^H w_{hk} \; b_{h}^{t}	
\end{equation}

Then, the final activation function also depends on the task involved. For connection 
temporal classification (CTC) tasks, such as sequence labeling or translation, it is common 
to use the softmax function~\cite{appendix:graves2006connectionist}. The CTC loss function is 
used, that is defined as the negative log probability of correctly labelling all examples in 
the training set.

The \textbf{backward pass} can be performed using a backpropagation through time (BPTT) 
algorithm~\cite{appendix:williams1995gradient, semPar:werbos1990}, which is an algorithm 
based on the standard backpropagation process but adapted to RNNs. The BPTT algorithm also 
consists of a repeated application of the chain rule, with the difference that, aside from 
the output layer, the loss function also depends on the activation of the hidden layer 
through its influence on the hidden layer at the next timestep. The derivatives can be 
calculated using the same procedure described in the Learning process subsection, but taking 
into consideration that the same weights are being reused at every timestep.

Since the classical RNN architecture only looks to past information, the \textbf{Bidirectional 
Recurrent Neural Network} (BRNN) was proposed to also include future 
context~\cite{appendix:SchusterP97}. The BRNN’s structure consists of two separate recurrent 
hidden layers, both connected to the same output layer. This idea allows a forwards and 
backwards training sequence, where the layer that performs the backward training receives the 
input sequence in the opposite direction. The output layer is not updated until both hidden 
layers have processed the entire input sequence.

\subsection{Long Short-Term Memory}
Though RNNs work well with short sentences, their performance decreases with long sentences 
that involve long term dependencies due to the \textbf{vanishing gradient} 
problem~\cite{seqlab:HochreiterS97, appendix:hochreiter2001gradient}, which occurs when the 
influence of the given input on the hidden layer (and therefore on the network output), 
either decays or explodes exponentially through the recurrent connections. In order to 
address this issue, the \textbf{Long Short-Term Memory} (LSTM) model~\cite{seqlab:HochreiterS97} 
was proposed. The LSTM model allows models to perform on tasks which require long range 
temporal dependencies, such as Sequence Labeling~\cite{seqlab:HuangXY15, seqlab:MaH16}, 
Machine Translation~\cite{nlToSparql:WuSCLNMKCGMKSJL16} or Summarization~\cite{appendix:MahasseniLT17}. 
As per RNNs, the LSTM also has a bidirectional variant, also known as 
BiLSTM~\cite{appendix:graves2005framewise,appendix:ChenC04a,appendix:ThireouR07}. 

An LSTM network is similar to a standard RNN, except that the summations units in the hidden 
layer are replaced by \textbf{memory blocks}, as shown in Figure~\ref{fig:oneCellLSTM}. This 
structure allows the memory cells to store and access information over long periods of time, 
thereby reducing the effects of the vanishing gradient problem.

\begin{figure}[!h]
    \centering
    \includegraphics[scale=.45]{imagenes/insertImage.png}
    \caption{LSTM memory block with one cell}
    \label{fig:oneCellLSTM}
\end{figure}

The following equations presented below are the formulas used to perform the forward pass 
evaluation over an LSTM with a single memory block for a timestep $t$. For multiple blocks 
the computations are repeated for each block. The values of $w_{ij}$, $a_j^t$ and $b_j^t$ 
have the same definition used before. 

First, the \textbf{Input Gates}, denoted as $a_\iota^t$~\ref{eq:inputHidden} and 
$b_\iota^t$~\ref{eq:inputActivation}. The number of inputs 
is denoted by $I$, the number of cells in the hidden layer is $H$ and the number of memory 
cells is $C$. The gate activation function $f$ most commonly used is the sigmoid function, so 
the gate activations are between 0 (gate closed) and 1 (gate open). The weight $w_{c\iota}$ 
represent the peephole weight from cell $c$ to the input gate. The state of the cell $c$ at 
time $t$ is denoted as $s_c^t$, which is calculated using Equation~\ref{eq:cellActivation}.

\begin{equation} \label{eq:inputHidden}
    a_\iota^t = \sum_{i=1}^I w_{i\iota} \; x_i^t + \sum_{h=1}^H w_{h\iota} \; b_h^{t-1} + \sum_{c=1}^C w_{c\iota} \; s_c^{t-1}
\end{equation}
\begin{equation} \label{eq:inputActivation}
    b_\iota^t = f(a_\iota^t)
\end{equation}

Then, the \textbf{Forget Gates}, denoted as $a_\phi^t$~\ref{eq:forgetHidden} and 
$b_\phi^t$~\ref{eq:forgetActivation}. The weight $w_{c\phi}$ represents the peephole weight 
from cell $c$ to the forget gate. Besides that, other symbols are equivalent to those 
mentioned for the input gate.

\begin{equation} \label{eq:forgetHidden}
    a_\phi^t = \sum_{i=1}^I w_{i\phi} \; x_i^t + \sum_{h=1}^H w_{h\phi} \; b_h^{t-1} + \sum_{c=1}^C w_{c\phi} \; s_c^{t-1}
\end{equation}
\begin{equation} \label{eq:forgetActivation}
    b_\phi^t = f(a_\phi^t)
\end{equation}

The \textbf{Cells}, denoted as $a_c^t$~\ref{eq:cellHidden} and $s_c^t$~\ref{eq:cellActivation}. The 
cell input activation function $g$ is usually hyperbolic tangent or sigmoid.

\begin{equation} \label{eq:cellHidden}
    a_c^t = \sum_{i=1}^I w_{ic} \; x_i^t + \sum_{h=1}^H w_{hc} \; b_h^{t-1}
\end{equation}
\begin{equation} \label{eq:cellActivation}
    s_c^t = b_\phi^t \; s_c^{t-1} + b_\iota^t \; g(a_c^t)
\end{equation}

Next, the \textbf{Outputs Gates}, denoted as $a_\omega^t$~\ref{eq:outputHidden} and 
$b_\omega^t$~\ref{eq:outputActivation}. The weight $w_{c\omega}$ represent the peephole w
eight from cell $c$ to the output gate.

\begin{equation} \label{eq:outputHidden}
    a_\omega^t = \sum_{i=1}^I w_{i\omega} \; x_i^t + \sum_{h=1}^H w_{h\omega} \; b_h^{t-1} + \sum_{c=1}^C w_{c\omega} \; s_c^{t-1}
\end{equation}
\begin{equation} \label{eq:outputActivation}
    b_\omega^t = f(a_\omega^t)
\end{equation}

Finally, the \textbf{Cell Outputs}, denoted as $b_c^t$~\ref{eq:cellOutput}. The cells outputs $b_c^t$ 
are the only ones connected to the other blocks in the layer. The index h is used to refer 
to cell outputs from other blocks in the hidden layer, if they exist. As per $g$, the output 
activation function $h$ is usually hyperbolic tangent or sigmoid, though sometimes the 
identity function can be used. 

\begin{equation} \label{eq:cellOutput}
    b_c^t = b_\omega^t \; f(s_c^t)
\end{equation}


\section{Convolutional Neural Networks}
\label{appendix:convolutionalNN}
The creation of \textbf{Convolutional Neural Networks} (CNNs) responds to the need to process 
certain types of data: images~\cite{appendix:OSheaN15}. Traditional ANNs do not perform well 
when processing images since its architecture does not properly support the computational 
complexity that involves processing large images as input. Whereas a 32$\times$32 image will 
be no problem to an traditional ANN, since it will require only 1024 parameters for a single 
neuron, images tend to have more resolution. On a higher scale, an image of 1024$\times$1024 
will instead require 1,048,576 parameters, which is a substantial increase. Moreover, if we 
consider colored images (RGB), a 1024$\times$1024 RGB image would require 3,145,728 
parameters. There is then a noticeable drawback of using standard feed-forward models where 
nodes are often connected to each node from the previous layer.

Convolutional Neural Networks share many similarities with standard ANNs in the way that both 
are composed of a set of neurons that are capable of learning, where each neuron receives an 
input and performs many operations, which commonly is a scalar product followed by an 
activation function. The difference resides in that CNNs are based on the idea that the input 
is shaped as an image. This idea allows the CNN architecture to adapt to this specific type 
of data. Then, a CNN architecture is built using three different types of layers: 
convolutional layers, pooling layers and fully-connected layers (same layers used in 
traditional ANNs). Additionally, each layer is organised into three dimensions: the spatial 
dimension (width and height) and the number of channels (also as known as depth, which is not 
the same as the number of layers). An example of a CNN architecture for pattern image 
classification is shown in Figure~\ref{fig:convNet}.

\begin{figure}[!h]
    \centering
    \includegraphics[scale=.45]{imagenes/insertImage.png}
    \caption{Convolutional Neural Network for pattern image classification.}
    \label{fig:convNet}
\end{figure}

\subsection{Convolutional layer}
A \textbf{convolution layer} is the central component of CNNs, which determines the output of 
neurons using calculations based on local regions of the input. This type of layer is based on 
learnable kernels, which define local convolution operations over the input vector. A 
convolution consists of the scalar product for each value in a kernel of dimension M$\times$N 
over a local region with the same size as the kernel used:

\begin{equation} \label{eq:convolution}
    (X \ast w)_{i, j} = \sum_{m=1}^M \sum_{n=1}^{N} X_{m, n} \cdot w_{i-m, j-n}
\end{equation}

\begin{figure}[!h]
    \centering
    \includegraphics[scale=.45]{imagenes/insertImage.png}
    \caption{Convolutional Neural Network for pattern image classification.}
    \label{fig:convNet}
\end{figure}

For example, in Figure~\ref{fig:convolutionPoolingExample} the convolution is being applied 
over a 3$\times$3 pooled vector, which is the size of the kernel $w$, that is, the local 
region of the entire input vector $X$. Though these kernels usually have a small spatial 
dimensionality, they are spread along the whole input vector. Besides kernel dimension, an 
application of padding over the input vector is also possible, which is the process of 
padding the border of the input with zeros. \textbf{Padding} controls both the dimensionality 
of the output volumes and gives more relevance to the input borders. Lastly, the 
\textbf{stride} is the amount of spaces the kernel is moved between each convolution. By 
setting a stride greater than 1, it is possible to reduce the amount of overlap and thus 
reduce the dimension of the activation output.

\begin{figure}[!h]
    \centering
    \includegraphics[scale=.45]{imagenes/insertImage.png}
    \caption{Convolution operation example.}
    \label{fig:convolutionPoolingExample}
\end{figure}

Let $N$ be the size of a input vector of size N$\times$N, $F$ the kernel F$\times$F dimension, 
$P$ the padding size, and $S$ the stride value; the final dimension of the output volume will 
be $\left\lfloor \frac{N - F + 2P}{S} + 1 \right\rfloor$. Note that, in the same way an image can be 
represented in 3 dimensions, the kernel can be extended to a third dimension by increasing 
the \textbf{number of channels}, thus giving control over the output depth of the 
convolutional layer. The number of channels, stride and padding are hyperparameters that can 
be optimised. 

Kernel values are the trainable parameters which a CNN can tuned in through the same type 
of learning process ANNs perfom. The composition of convolutional layers allows to reduce the 
dimensionality of a Neural Network in terms of learnable parameters, based on the assumption 
of parameter sharing. This assumption says that \textit{“if one region is useful to compute at 
a set spatial region, then it is likely to be useful in another region”}. The constraints of 
each activation within the output volume to the same weights and bias means a significant 
decrease in the number of parameters used in a convolution layer. Then, the backward pass for 
each neuron in the output represents the overall gradient across channels, where only a 
single set of weights is updated.

After the application of the convolution, an activation function is applied over the output 
volume. The most common choice is the rectified linear unit (ReLu) which is an elementwise 
function that given a certain threshold $\alpha$ will set all values less than $\alpha$ to 0 .

\subsection{Pooling layer}
A pooling layer aims to reduce the volume of an input representation with the purposes of 
reducing the computational complexity of the model. After each activation from a convolutional 
layer, a pooling layer can be applied to scale its dimensionality through a reducing function. 
The most common type of pooling is the max-pooling layer, which is a kernel that applies a 
MAX reduction on a local region as per a convolutional kernel. Other examples are average 
pooling, or general pooling that applies average reduction and $L_1$/$L_2$ normalization 
respectively.

Though pooling layers also include settings such as kernel size, stride or number of channels, 
they do not add learnable parameters. However, they do influences in the backward pass to 
calculate the gradients. Lastly, it is recommended to keep a low stride and kernel size since 
its application could affect negatively performance if large values are used.

\subsection{Common architectures}
As mentioned before, a CNN architecture is commonly built with an input layer, which receives 
the values of the image, followed by various stacked convolutional layers, each one followed 
by pooling layers, and a final stack of fully-connected layers. However, defining exactly the 
amount of layers to use is not a simple task. In fact, most of the literature is based on 
standard architectures that have shown good results on certain image processing tasks.

Among the most popular architectures, ImageNet~\cite{appendix:KrizhevskySH12} is a Deep 
Convolutional Network with five convolutional layers, some followed by max-pooling layers, 
followed by two fully-connected layers. Another example is ResNet~\cite{appendix:HeZRS16}, 
which includes \textbf{residual connections} that aim to reduce the vanishing gradient 
problem that a very dense convolutional network can suffer. The main principle of residual 
connections is to create connections between non-adjacent layers that skip a certain 
amount of layers.
\chapter{Question Answering Dataset}
\label{appendix:qaDataset}
We provide more details on the data used for the training and validation of our system.

\section{Normalized Dataset Format}
\label{appendix:qaDataset/normalizedFormat}
As mentioned in the \textit{Experimental Design} chapter~\ref{cap4:experimentalDesign}, we proposed 
a dataset format so the process 
of generating the Query Template dataset and the Sequence Labeling datasets could be simplified. This 
normalized format is also used for the other test datasets (QALD-7 and WikiSPARQL). In 
Listing~\ref{lst:normalizedDatasetExample} we can see an example of what information each case contains:

\begin{sparqlcode}[%
    caption={Example of one \LCQuADtwo{} case following our proposed normalized format.}, 
    label={lst:normalizedDatasetExample}]
    "question_id": 30226,
    "natural_language_question": "Did Alexander Hamilton practice law?",
    "query_answer": [
        {
            "query_id": 0,
            "sparql_query": 
                "ASK WHERE { wd:Q178903 wdt:P106 wd:Q40348 }",
            "entities": [ 
                {"label": "Alexander Hamilton", "entity": "wd:Q178903"},
                {"label": "lawyer", "entity": "wd:Q40348"}
            ],
            "slots": [
                {"slot": "<sbj_1>", "label": "Alexander Hamilton"},
                {"slot": "<obj_1>", "label": "lawyer"}
            ],
            "sparql_template": "ASK WHERE { <sbj_1> wdt:P106 <obj_1> }"
        }
    ]
\end{sparqlcode}

\begin{itemize}
    \item \textbf{Question ID}: the unique identifier of the question; for \LCQuADtwo{} we decided to 
    treat each verbalized and paraphrased version of the normalized question as a separate case. 
    Verbalized cases are then numbered from $0$ to $30,225$, and paraphrased cases from $30,226$ to 
    $60,451$.
    \item \textbf{Natural Language Question}: the question text (verbalized or paraphrased version).
    \item \textbf{Query Answer}: a list of possible SPARQL query valid responses. We allow more than 
    one possible query since for each question there are many ways to reach the expected answer (for 
    example, \dquotesit{What is the biggest country?} may refer to population, area, etc.; however in 
    this work we only have one SPARQL query per case). Each query answer case has the following fields:
    \begin{itemize}
        \item \textbf{Query ID}: unique identifier to identify query answers for the same question.
        \item \textbf{SPARQL query}: the SPARQL query string.
        \item \textbf{Entities}: list of expected annotations of the entities being used in the 
        SPARQL query answer. Each annotation contains the entity URL and the label of the question 
        associated with that entity.
        \item \textbf{Slots}: list of expected slots to use for the Slot Filling system. Each slot 
        contains the associated label in the question and its corresponding placeholder in the Query 
        Template.
        \item \textbf{SPARQL Template}: the Query Template derived from the SPARQL query.
    \end{itemize}
\end{itemize}

\section{LC-QuAD 2 base templates}
\label{appendix:qaDataset/baseTemplates}
In the \textit{Results} chapter~\ref{cap5:results} we displayed an analysis based on the 22 base 
templates used for building 
the \LCQuADtwo{} dataset~\cite{dataset:lcquad2-DubeyBA019}. According to each base template structure, 
we established which cases are considered \textbf{complex cases}, which allows us to estimate the 
percentage of complex questions this dataset contains. A base template is considered a \textit{complex 
case} if it includes any of the following operations: (1) counting, (2) filtering, (3) ordering, 
(4) use of strings or numbers,(5) use of property statements, or (6) returns more than one variable. 

Next, we present a brief overview of each one of these base templates along with a representative 
example from the dataset:

\begin{enumerate}
    \item \textbf{ask\_one\_fact}: ASK query with one query triple.
    \begin{itemize}
        \item Example: \textit{Did Alexander Hamilton practice law?}
        \item Query:\\
        \mbox{}\\
        \begin{sparqlcode}[]
ASK WHERE { 
    wd:Q178903 wdt:P106 wd:Q40348 
}
        \end{sparqlcode}
    \end{itemize}

    \item \textbf{ask\_one\_fact\_with\_filter}: ASK query with one query triple and a numeric filter 
    operation (either less than, equal, or greater than).
    \begin{itemize}
        \item Example: \textit{Does the standard molar entropy of silver equal 34.08?}
        \item Query:\\
        \mbox{}\\
        \begin{sparqlcode}[]
ASK WHERE { 
    wd:Q1090 wdt:P3071 ?obj filter(?obj = 34.08) 
}
        \end{sparqlcode}
    \end{itemize}

    \item \textbf{ask\_two\_facts}: ASK query with two query triples. Note that the same subject 
    and property are used in both query triples.
    \begin{itemize}
        \item Example: \textit{Was William Henry Harrison both a United States senator and Governor of 
        Indiana?}
        \item Query:\\
        \mbox{}\\
        \begin{sparqlcode}[]
ASK WHERE { 
    wd:Q11869 wdt:P39 wd:Q16147601 . 
    wd:Q11869 wdt:P39 wd:Q13217683 
}
        \end{sparqlcode}
    \end{itemize}

    \item \textbf{count\_one\_fact\_object}: SELECT query with COUNT operation from one query triple. 
    Count the number of objects that match the query triple.
    \begin{itemize}
        \item Example: \textit{How many Latin conjugations are there?}
        \item Query:\\
        \mbox{}\\
        \begin{sparqlcode}[]
SELECT (COUNT(?obj) AS ?value ) WHERE { 
    wd:Q397 wdt:P5206 ?obj 
}
        \end{sparqlcode}
    \end{itemize}

    \item \textbf{count\_one\_fact\_subject}: SELECT query with COUNT operation from one query triple. 
    Count the number of subjects that match the query triple.
    \begin{itemize}
        \item Example: \textit{What is the number of spore print colors for olive?}
        \item Query:\\
        \mbox{}\\
        \begin{sparqlcode}[]
SELECT (COUNT(?sbj) AS ?value ) WHERE { 
    ?sbj wdt:P787 wd:Q864152 
}
        \end{sparqlcode}
    \end{itemize}

    \item \textbf{rank\_instance\_of\_type\_one\_fact}: SELECT query with two query triples with 
    sorting and limit. The first query triple always uses the property instance of (wdt:P31). 
    Ordering might vary (ascending or descending).
    \begin{itemize}
        \item Example: \textit{What battery power station has the highest amount of energy storage 
        capacity?}
        \item Query:\\
        \mbox{}\\
        \begin{sparqlcode}[]
SELECT ?ent WHERE { 
    ?ent wdt:P31 wd:Q810924 . 
    ?ent wdt:P4140 ?obj 
} ORDER BY DESC(?obj) LIMIT 5
        \end{sparqlcode}
    \end{itemize}
    
    \item \textbf{rank\_max\_instance\_of\_type\_two\_facts}: SELECT query with three query triples using 
    sorting and limit. The first query triple always uses the property instance of (wdt:P31). 
    Descending ordering is used to get the maximum value.
    \begin{itemize}
        \item Example: \textit{What is the largest village in Muchinigi Puttu?}
        \item Query:\\
        \mbox{}\\
        \begin{sparqlcode}[]
SELECT ?ent WHERE { 
    ?ent wdt:P31 wd:Q532 . 
    ?ent wdt:P2046 ?obj . 
    ?ent wdt:P131 wd:Q11107378 
} ORDER BY DESC(?obj) LIMIT 5
        \end{sparqlcode}
    \end{itemize}
    
    \item \textbf{rank\_min\_instance\_of\_type\_two\_facts}: Same as the previous base template but with 
    ascending ordering to get the minimum value.
    \begin{itemize}
        \item Example: \textit{What is the name of a manned spacecraft in low Earth orbit with 
        smallest periapsis?}
        \item Query:\\
        \mbox{}\\
        \begin{sparqlcode}[]
SELECT ?ent WHERE { 
    ?ent wdt:P31 wd:Q7217761 . 
    ?ent wdt:P2244 ?obj . 
    ?ent wdt:P522 wd:Q663611
} ORDER BY ASC(?obj) LIMIT 5
        \end{sparqlcode}
    \end{itemize}
    
    \item \textbf{select\_object\_instance\_of\_type}: SELECT query with two query triples. The first 
    query triple always uses the property instance of (wdt:P31). Return the entities that are the 
    object of the first query triple.
    \begin{itemize}
        \item Example: \textit{What is the prefecture of Hiroshima in Japan?}
        \item Query:\\
        \mbox{}\\
        \begin{sparqlcode}[]
SELECT DISTINCT ?obj WHERE { 
    wd:Q34664 wdt:P131 ?obj . 
    ?obj wdt:P31 wd:Q50337 
}
        \end{sparqlcode}
    \end{itemize}
    
    \item \textbf{select\_object\_using\_one\_statement\_property}: SELECT query with three query 
    triples. Uses one property statement and one property qualifier. 
    \begin{itemize}
        \item Example: \textit{When did Louis XVIII of France, husband of Marie Josephine of Savoy, 
        die?}
        \item Query:\\
        \mbox{}\\
        \begin{sparqlcode}[]
SELECT ?value WHERE { 
    wd:Q7750 p:P26 ?s . 
    ?s ps:P26 wd:Q231844 . 
    ?s pq:P582 ?value
}
        \end{sparqlcode}
    \end{itemize}
    
    \item \textbf{select\_one\_fact\_object}: SELECT query with one query triple. Return the entities 
    that are the subject of that query triple.
    \begin{itemize}
        \item Example: \textit{Which are the characters for La Malinche?}
        \item Query:\\
        \mbox{}\\
        \begin{sparqlcode}[]
SELECT DISTINCT ?answer WHERE { 
    ?answer wdt:P674 wd:Q230314
}
        \end{sparqlcode}
    \end{itemize}
    
    \item \textbf{select\_one\_fact\_subject}: SELECT query with one query triple. Return the entities 
    that are the object of that query triple.
    \begin{itemize}
        \item Example: \textit{Which is the electric charge for antihydrogen?}
        \item Query:\\
        \mbox{}\\
        \begin{sparqlcode}[]
SELECT DISTINCT ?answer WHERE { 
    wd:Q216121 wdt:P2200 ?answer
}
        \end{sparqlcode}
    \end{itemize}
    
    \item \textbf{select\_one\_qualifier\_value\_and\_object\_using\_one\_statement\_property}: SELECT query 
    with three query triples. Uses one property statement and one property qualifier. Only one entity 
    used for the first query triple.
    \begin{itemize}
        \item Example: \textit{What grant did Konrad Lorenz win, and who won it with him?}
        \item Query:\\
        \mbox{}\\
        \begin{sparqlcode}[]
SELECT ?value1 ?obj WHERE { 
    wd:Q78496 p:P166 ?s . 
    ?s ps:P166 ?obj . 
    ?s pq:P1706 ?value1 . 
}
        \end{sparqlcode}
    \end{itemize}
    
    \item \textbf{select\_one\_qualifier\_value\_using\_one\_statement\_property}: SELECT query with three 
    query triples. Uses one property statement and one property qualifier. Two entities used for 
    building each query, where the second entity was used as the object of the property qualifier 
    triple.
    \begin{itemize}
        \item Example: \textit{What role did Theodore Roosevelt occupy after William McKinley?}
        \item Query:\\
        \mbox{}\\
        \begin{sparqlcode}[]
SELECT ?obj WHERE { 
    wd:Q33866 p:P39 ?s . 
    ?s ps:P39 ?obj . 
    ?s pq:P1365 wd:Q35041 
}
        \end{sparqlcode}
    \end{itemize}
    
    \item \textbf{select\_subject\_instance\_of\_type}: SELECT query with two query triples. The first 
    query triple always uses the property instance of (wdt:P31). Return the entities that are the 
    subject of the first query triple.
    \begin{itemize}
        \item Example: \textit{Which irresistible illness is caused by Staphylococcus aureus?}
        \item Query:\\
        \mbox{}\\
        \begin{sparqlcode}[]
SELECT DISTINCT ?sbj WHERE { 
    ?sbj wdt:P828 wd:Q188121 . 
    ?sbj wdt:P31 wd:Q18123741 
}
        \end{sparqlcode}
    \end{itemize}
    
    \item \textbf{select\_subject\_instance\_of\_type\_contains\_word}: SELECT query with two query triples. 
    The first query triple always uses the property instance of (wdt:P31). The second query triple 
    also includes a filter over the string value to check whether the target word is contained in 
    this string (includes filtering for the English language).
    \begin{itemize}
        \item Example: \textit{What is the title of a human that contains the word vitellius in their 
        name.}
        \item Query:\\
        \mbox{}\\
        \begin{sparqlcode}[]
SELECT DISTINCT ?sbj ?sbj_label WHERE { 
    ?sbj wdt:P31 wd:Q5 . 
    ?sbj rdfs:label ?sbj_label . 
    FILTER(CONTAINS(lcase(?sbj_label), 'vitellius')) . 
    FILTER (lang(?sbj_label) = 'en') 
} LIMIT 25
        \end{sparqlcode}
    \end{itemize}
    
    \item \textbf{select\_subject\_instance\_of\_type\_starts\_with}: SELECT query with two query triples. 
    The first query triple always uses the property instance of (wdt:P31). The second query triples 
    also includes a filtering over the string value to check whether this string starts with a target 
    letter (including filtering for the  English language).
    \begin{itemize}
        \item Example: \textit{What are the video games which start with the letter W?}
        \item Query:\\
        \mbox{}\\
        \begin{sparqlcode}[]
SELECT DISTINCT ?sbj ?sbj_label WHERE { 
    ?sbj wdt:P31 wd:Q7058673 . 
    ?sbj rdfs:label ?sbj_label . 
    FILTER(STRSTARTS(lcase(?sbj_label), 'w')) . 
    FILTER (lang(?sbj_label) = 'en') 
} LIMIT 25
        \end{sparqlcode}
    \end{itemize}
    
    \item \textbf{select\_two\_answers}: SELECT query with two query triples. Return two variables 
    (also known as double intention), with each one being the object of each query triple.
    \begin{itemize}
        \item Example: \textit{Where was Jane Austen born and where did she die?}
        \item Query:\\
        \mbox{}\\
        \begin{sparqlcode}[]
SELECT ?ans_1 ?ans_2 WHERE { 
    wd:Q36322 wdt:P19 ?ans_1 . 
    wd:Q36322 wdt:P20 ?ans_2 
}
        \end{sparqlcode}
    \end{itemize}
    
    \item \textbf{select\_two\_facts\_left\_subject}: SELECT query with two query triples.
    \begin{itemize}
        \item Example: \textit{Who were the creators of The Late Awesome Planet Soil?}
        \item Query:\\
        \mbox{}\\
        \begin{sparqlcode}[]
SELECT ?answer WHERE { 
    wd:Q22081649 wdt:P144 ?obj . 
    ?obj wdt:P50 ?answer
}
        \end{sparqlcode}
    \end{itemize}
    
    \item \textbf{select\_two\_facts\_right\_subject}: SELECT query with two query triples. Similar to 
    the previous template (in fact we could not find any difference between both cases).
    \begin{itemize}
        \item Example: \textit{What time zone is Arizona State University in?}
        \item Query:\\
        \mbox{}\\
        \begin{sparqlcode}[]
SELECT ?answer WHERE { 
    wd:Q670897 wdt:P17 ?obj . 
    ?obj wdt:P421 ?answer
}
        \end{sparqlcode}
    \end{itemize}
    
    \item \textbf{select\_two\_facts\_subject\_object}: SELECT query with two query triples. Uses one 
    entity in the subject of the first query triple, and another one in the object of the second 
    query triple.
    \begin{itemize}
        \item Example: \textit{What lake of Sao Jorge island has the tributary Curoca River?}
        \item Query:\\
        \mbox{}\\
        \begin{sparqlcode}[]
SELECT ?answer WHERE { 
    wd:Q743362 wdt:P206 ?answer . 
    ?answer wdt:P974 wd:Q10361834
}
        \end{sparqlcode}
    \end{itemize}
    
    \item \textbf{select\_two\_qualifier\_values\_using\_one\_statement\_property}: SELECT query 
    with four query triples. Uses one property statement and two property qualifiers. Return the two object 
    values of each query triple using a property qualifier.
    \begin{itemize}
        \item Example: \textit{Give me the year and name of the person with whom Bob Barker shared the 
        MTV Movie Award for Best Fight.}
        \item Query:\\
        \mbox{}\\
        \begin{sparqlcode}[]
SELECT ?value1 ?value2 WHERE { 
    wd:Q381178 p:P166 ?s . 
    ?s ps:P166 wd:Q734036 . 
    ?s pq:P585 ?value1 . 
    ?s pq:P1706 ?value2 
}
        \end{sparqlcode}
    \end{itemize}
    
\end{enumerate}
\end{appendices}
 % opcionales

\bibliographystyle{plain}
\bibliography{bibliografia}

\end{document}
