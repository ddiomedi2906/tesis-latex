\section{Hypothesis}
In this work, we propose the following hypothesis: \dquotesit{a combination of Information 
Extraction with Semantic Parsing can develop a Question-Answering system that outperforms 
a system that relies only on Semantic Parsing in the Question Answering over Knowledge 
Graphs task}.

In order to measure performance, this work will use metrics based on two perspectives: 
one focuses on the final answers that are derived from Question Answering over Knowledge 
Graphs (KGQA) benchmarks (i.e., a Question Answering-based evaluation), and the second 
focuses on how close is the generated \SPARQL{} query compared with the expected query (i.e., 
a Machine Translation-based evaluation).

The scope of this work will be limited to answering questions in English, but similar 
techniques should be applicable in any other language assuming the availability of similar 
datasets for that language. In the same way, this hypothesis will be explored in the context of 
questions over Wikidata, so the results might differ for other Knowledge Graphs. Nevertheless, 
the selected approach should be generalizable to other domains. 

\section{Objectives}
\subsection*{General Objective}
% \lipsum[1-3]
We aim to improve upon state-of-art Question Answering systems 
based on Neural Semantic Parsing models by reducing vocabulary dependency on the 
data used in the learning process of such models.
\subsection*{Specific Objectives}
% \lipsum[1-3]
The specific objective is to build a Question-Answering system over Wikidata in English, 
that relies on Entity Linking and Neural Machine Translation systems, with an intermediate 
system that combines both tools. Our initial claim is that such a system can improve upon 
the state-of-the-art Neural Machine Translation approaches found in the literature.  

\section{Methodology}
Accomplishing the proposed objectives involves the following tasks:
\begin{itemize}
    \item Survey previous work regarding Neural Machine Translation, Entity Linking, and 
    Question Answering approaches that rely on Neural Machine Translation.
    \item Define a benchmark that should include KGQA datasets for training, validation and 
    testing along with metrics to compare all involved systems.
    \item Define a baseline system for Question Answering based on Neural Machine Translation.
    \item Define a pipeline process to convert a natural language question into a \SPARQL{} query 
    by combining Entity Linking techniques with Neural Machine Translation.
    \item Implement a Question-Answering system over Wikidata in English based on the designed pipeline.
    \item Validate the proposed system by comparing it with baseline approaches over the 
    proposed benchmark.
\end{itemize}
