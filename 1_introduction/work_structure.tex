\section{Work Structure}
This worked is divided into the following chapters:

\begin{enumerate}
    \item In Chapter~\ref{cap2:theoFrame}, we describe the theoretical framework 
    enclosed on this work. This chapter covers concepts about what is the Semantic 
    Web, Information Extraction methods and how they relate with Semantic Web 
    technologies, Semantic Parsing applied on translating natural language to 
    \SPARQL, and the current state and challenges of the Question Answering over 
    Knowledge Graphs task.
    \item In Chapter~\ref{cap3:system}, we give an overview of the proposed Question 
    Answering system for this work. This includes a general explanation on the 
    pipeline proposed to generate a \SPARQL{} query, and more specific details on how 
    each component is designed.    
    \item In Chapter~\ref{cap4:experimentalDesign}, we go into details about the 
    experiments we run on this work. We present the research questions we aimed to answer, 
    the baseline we compare our system with, and the metrics used to quantify the 
    performance of each system.    
    \item In Chapter~\ref{cap5:results}, we present the results derived from running the 
    proposed experiments. Aside from that, we include a brief discussion and analysis of 
    the results.    
    \item In Chapter~\ref{cap6:conclusions}, we summarize the conclusion of this work, 
    discuss its limitations and the future work regarding Question Answering over 
    Knowledge Graphs.
    
\end{enumerate}
