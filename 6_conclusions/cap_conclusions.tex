\chapter{Conclusions}
\label{cap6:conclusions}
In this work, we address the Question Answering over Knowledge Graphs task, which, given a natural 
language question, consists of generating the expected answer when evaluated on the KG. In particular, 
we proposed a Question Answering system that generates a \SPARQL{} query expressing the question, and that 
combines Entity Linking systems with a Neural Machine Translation model to address the dependency on the 
vocabulary used for training the NMT model when generating \SPARQL{} queries.

The \SPARQL{} query generation process was divided into several stages. First, we trained a 
Convolutional Sequence to Sequence model and we used it to generate a Query Template, which is a 
\SPARQL{} query with its entities replaced by placeholders. Secondly, relevant entities were extracted 
using an Ensemble Entity Linking system, which combines many Individual Entity Linking systems to 
increase the amount of entities identified. Two variants of Ensemble Entity Linking system were 
proposed: a Precision Priority system and a Voting system. Finally, we proposed a Slot Filling system 
to fill entities into the Query Template, which relies on a Sequence Tagger (based on the 
state-of-art in Named Entity Recognition) and a proposed filling algorithm, giving the final \SPARQL{} 
query as a result.

We conducted various experiments to compare the performance of our system with a proposed baseline, 
which includes an overall comparison for the tasks of \SPARQL{} query generation and Question Answering, 
and also an analysis for each one of the three sub-tasks involved in our proposal (Query Template 
Generation, Entity Linking, and Slot Filling).

According to the \textbf{hypothesis} proposed in this work, we showed that by combining Entity 
Linking and Neural Semantic Parsing through a Slot Filling system, it is possible to obtain better 
performance in the Question Answering over Knowledge Graph task. This is according to the proposed 
metrics used to compare the system: BLEU score and Accuracy for \SPARQL{} query generation, and Precision, 
Recall and F1 score for Question Answering. We found that this increase in performance comes mainly 
from the reduction of errors in the system due to not identifying the correct entities required to 
build the \SPARQL{} query.

It is important to remember that our hypothesis was explored in the context of Wikidata for questions 
in the English language. Nevertheless, the approach we follow should be generalizable to other 
Knowledge Graphs and languages. The conclusions based on the hypothesis also allow us to determine 
that our \textbf{specific objective} of building a Question Answering system over Wikidata for 
English questions was accomplished, though this system is obviously subject to further improvement 
in future.

At the time that this work was developed, and to the best of our knowledge, the models described by 
Yin et al.~\cite{nmt:nl-to-sparql-Yin19} were the state-of-the-art for KGQA systems based on Neural 
Machine Translation models. They validate each model over DBpedia datasets. In our case, we took the 
model they stated has better performance to compare with our system. Moreover, we validate our 
results using \LCQuADtwo{} (a more complex dataset in terms of questions variations and \SPARQL{} query 
complexity); \QALDseven{} (which is also used to evaluate Information Retrieval-based QA systems), and 
\WikiSPARQL{} (a new dataset with cases that require new types of operations). That being said, we 
think our \textbf{general objective} of improving the state-of-the-art was accomplished.

The last thing to mention are the \textbf{limitations} we identified in our work. The first issue is 
related with the performance shown in the Query Template generation task. According to our results, 
most of the cases where the \SPARQL{} query was built incorrectly were due to the Query Template being 
incorrect. Even though in this work we could not identify the main cause of the poor performance for 
this task, we think it might be caused by different reasons: either the \LCQuADtwo{} dataset contained 
cases that are inherently difficult to respond even for an \SPARQL{} expert, or it does not contain 
enough cases to allow the NMT model to generalize across templates. As a consequence, the \SPARQL{} 
queries our proposed KGQA system can generate are strongly limited by the type of question found in 
the \LCQuADtwo{} training set. For example, our system may be able to generate questions that require 
counting, but given that the only COUNT type query included in the \LCQuADtwo{} dataset uses only one 
fact, questions that require counting and use more than one fact will not be correctly interpreted.

The second limitation involves our proposed Slot Filling system using a Sequence Tagger to label the 
placeholders in the Query Template, which is an \dquotes{ad-hoc} solution for this work. This means that 
if we want to train our proposal over another dataset, we will have to adapt that dataset to be utilized 
to train the Sequence Tagger. That is one of the reasons we also proposed a normalized dataset format, 
so the Sequence Tagger dataset construction could be simplified.

\section{Relevance and Contributions}
\label{cap6:conclusions/relevanceContributions}
First, we propose a benchmark for comparing Question Answering systems that work over Wikidata. By 
cleaning the \LCQuADtwo{} to have less noisy cases and a normalized format, we reduce the amount of 
work others will have to spend in having a ready to use dataset to either train Neural-based models 
or evaluating their KGQA systems. Moreover, we deliver a new dataset that includes new \SPARQL{} 
operations and patterns. These datasets can not only be used to evaluate KGQA systems, but also to 
compare Entity Linking systems and Slot Filling systems.

Second, we contribute with a new Question Answering system over Wikidata for questions in English. 
The implementation comes from our proposed three-stage pipeline that addresses the \SPARQL{} Query 
generation process. This pipeline can be used to implement Question Answering systems focused on 
other Knowledge Graphs or languages. 

As a side effect, we propose an approach to boost the performance of Entity Linking systems when 
dealing with questions that have little to no context: Ensemble Entity Linking systems. We proposed 
two variants on how to combine results from individual Entity Linking systems to deliver a more 
complete response. The Precision priority system shows how to improve performance by prioritizing 
results from more precise systems. 

The type of questions our system can address are directly tied to the types of questions contained in 
the \LCQuADtwo{} dataset. Some examples are boolean questions (including cases that use literal numeric 
values), and questions that require use of string operations, counting, property statements, or 
ranking. Even though not all query variants are considered, this is a big step forward in terms of 
the type of queries that are supported, since none of the previous Neural-based models support cases 
we did consider by using the \LCQuADtwo{} dataset (e.g. use of numeric values or literal strings). 

In terms of overall relevance, we deliver new insights on how to address the Question Answering over 
Knowledge Graphs task. In particular, we proposed an approach that aims to address the vocabulary 
dependency that Neural-based models might suffer, and show how there is a direct impact on 
Neural-based KGQA systems’ performance. 

\section{Future work}
\label{cap6:conclusions/futureWork}
The Question Answering over Knowledge Graphs task is still a challenging problem despite there being 
plenty of work in this field. There are individual components of our system that can be improved but 
also new things to try or add to our system.

One thing is to work on improving the training dataset by increasing its size and adding the \SPARQL{} 
query variants the \LCQuADtwo{} dataset lacks. We think a bigger dataset would help the Query Template 
generator model have a better learning process, and the addition of new variants would let the model 
address a wider range of questions. Furthermore, we should focus on questions that are asked more 
frequently by the users of Wikidata. Then, a possible next step would be to implement a crowdsourcing 
tool that lets users ask questions for our system, thus identifying common patterns or categories 
that can be materialized in new query base templates. Lastly, it would be interesting to add new 
operations in the training examples (e.g. property paths, conditional operations, optional patterns 
or unions of results).

Another aspect that could be improved is the learning architecture we are using for the Query 
Template generator, where we could also try either of the other models used in Yin et al.'s work, or 
try new architectures from the state-of-the-art for Semantic Parsing or Machine Translation. One 
other component which might help better estimate the model performance in practice, would be to 
follow a k-fold cross-validation during the training phase~\cite{futureWork:anguita2012k}. Aside 
from that, we could also propose new logical forms for representing Query Templates. For example, we 
can condense some common query patterns (e.g. the syntax pattern to see whether a string is contained 
in an entity label is always the same), or use program-based representations. 

Even though our proposed Slot Filling system manages to perform well enough, we would like to have a 
system that reduces the amount of incorrect slots as much as possible. One proposal would be to 
test other placeholder classifications that capture different dimensions of entities, e.g. tag 
entities as humans, organizations, places, etc. Another approach that we may 
follow is to include the placeholder labeling during the Query Template generation stage, perhaps by 
adapting the Pointer Network architecture~\cite{key:vinyals2015pointer}, which 
aims to address the issue of vocabulary dependency by adding pointers in the output sequence that 
map to labels contained in the text being processed.

To work more on the Entity Linking stage, we see a potential in the Voting Entity Linking system we 
proposed in this work. We think that the performance of this system can be improved horizontally, 
i.e., we can add further Individual Entity Linking systems to improve the results of the voting scheme. 
Moreover, we can add Entity Linking systems focused on specific categories (e.g. sports, medicine, 
music, etc), so more rare entities can be identified more easily.

As we mentioned before, our approach should be generalizable to other languages or Knowledge Graph 
domains. Another possible step is to evaluate our system over other languages such as Spanish, 
French, or German, or to try to generate \SPARQL{} queries for other Knowledge Graphs such as DBpedia. 
To give an idea, the main changes that would be required are essentially to provide a dataset that either 
changes the input language of the questions or the input Knowledge Graph and \SPARQL{} queries, used 
for both training and evaluation.
