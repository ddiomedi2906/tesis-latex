\chapter{Experimental Design}
\label{cap4:experimentalDesign}

\section{Question Answering general overview}
\label{cap4:experimentalDesign/QaOverview}
The task of Question Answering over Knowledge Graphs is far from being resolved, where one of 
the main challenges that still remains is answering complex questions. Current approaches rely 
mostly on hand-crafted rules, achieving decent performance over simple questions (that only 
require a few query triple patterns), but not for more complex questions. Complex questions are 
difficult to handle mainly because they may be structured in a more elaborate way, and most of 
them require the use of more advanced SPARQL operations (such as aggregations, ranking, or the 
use of more complex graph patterns).

Some recent approaches based on Neural Networks have shown some potential for handling more 
complex questions. These Neural-based approaches usually aim to build the entire SPARQL query 
given a natural question, reducing the problem of Question Answering to Semantic Parsing. The 
main weakness of these systems is that they are dependent on the vocabulary of the data used 
for training such models, so even though these systems can generalize to different question 
forms, they are not able to identify unknown entities that were included in the training set.

In order to address this problem, we study the effectiveness of an approach that combines Neural 
Semantic Parsing with Entity Linking through a Slot Filling method proposed in this work. The 
research questions we propose are the following:

\begin{itemize}
    \item \textit{Can Entity Linking and Slot Filling improve the performance of Question 
    Answering systems over Knowledge Graphs based on Neural Semantic Parsing?}
    \item \textit{Which stages of the proposed Question Answering pipeline (Entity Linking, 
    Query Template Generation, Slot Filling) produce the most errors, what types of errors do 
    they produce, and how do they affect the overall performance of the Question Answering system?}
\end{itemize}

Note that all these questions are addressed specifically in the context of Wikidata as 
the target Knowledge Graph and for questions in English.

\section{Question Answering Dataset}
\label{cap4:experimentalDesign/QaDataset}
In this section we describe the dataset used for this work which includes the preprocessing 
required to improve the quality of the data used for training, validation and testing all of 
the modules (Query Generation, Entity Linking, and Slot Filling) and the final end-to-end 
Question Answering System. 

First, we describe the revision and cleaning performed over the LC-QuAD 2 dataset, which is the 
main source we use to train the Neural Network-based models. Second, we explain the SPARQL query 
encoding used for the training of the Query Template generation and the baseline SPARQL Query 
Generation models. Third, we describe the process for generating the Query Templates used for 
training the Query Template generation model. Fourth, we describe the same process for the Slot 
Filling dataset. Finally, we present the normalized dataset format which we designed to handle 
all the datasets used.

\subsection{LC-QuAD 2 Dataset Cleaning}
\label{cap4:experimentalDesign/QaDataset/cleaning}
The LC-QuAD 2.0 dataset requires a cleaning process where some cases were discarded or fixed due 
to various issues in either their question or SPARQL query answer. First we mention which cases 
were discarded according to some problems found in their paraphrased or verbalized question. 
Next, we describe which cases were discarded according to problems found in their SPARQL query. 
Since there are some queries with an easy fix, we explain briefly those cases. 

Given that the questions contained in the LC-QuAD 2.0 dataset were built from a two-step 
question correction (a verbalization and a later paraphrasing), each dataset case contains 
three representations of the same question: a normalized question, a verbalized question and a 
paraphrased question. An example is shown in Listing~\ref{lst:questionExampleLcquad2}. Since 
this process was performed using a crowdsourcing tool, we found issues related with question 
verbalization (e.g., the human that performed the verbalization didn’t understand the normalized 
question) or loss of the question’s semantics through the conversion from normalized to 
paraphrased questions.

\begin{lstlisting}[captionpos=b, 
    caption=Example of questions contained in one case of the LC-QuAD 2 dataset., 
    label=lst:questionExampleLcquad2,
    basicstyle=\ttfamily,frame=single]
normalized question  -> Did {Alexander_Hamilton} {occupation} {lawyer}?
verbalized question  -> Is Alexander Hamilton a lawyer?
paraphrased question -> Did Alexander Hamilton practice law?
\end{lstlisting}

Since some cases may have issues on either the verbalized or the paraphrased question, we 
decided to maintain a one-to-one correspondence between question and query. Although it would 
be interesting to test whether having many variations of the same question might affect the 
system performance, we think that experiment has to be done with a dataset where each case has 
the same amount of variations (which cannot be guaranteed with this dataset without having to 
discard an uncertain amount of valid cases). Then, in each case we first use the paraphrased 
question as the case’s question. However, if the paraphrased question is not valid, we use the 
verbalised question instead (if it is valid). A case is discarded if both the verbalised and 
the paraphrased questions present an issue.

To define if a question is an invalid verbalisation/paraphrase, the following issues are taken 
into account:

\begin{enumerate}
    \item Question is null, empty or not applicable (marked as “NA”). 
    \item Question has not been corrected, e.g., it still contains brackets from the normalised 
    question, or it is the same question as the normalised version without brackets.
    \item The question contains the answer to the previous normalised or verbalised question 
    instead of the verbalisation or paraphrasing, respectively. 
    \item The length of the actual version compared with the previous version is too long or too 
    short, which for us is a way to identify issue 3 or an indicator of a question that did not 
    aintain the semantic meaning of the previous version.
    \item The question is too short or too long. We set a minimum of 4 tokens and a maximum of 
    30, where each token is divided by whitespace characters.
\end{enumerate}

Note that given the large number of questions in the dataset, this validation was not performed 
manually but rather using a script to identify these issues as best as possible. Thus some cases 
may slip into the final preprocessed dataset, though we consider that as an acceptable amount of 
noise.

Other cases presented issues with their Wikidata SPARQL query, where either the query was 
empty/null, or contained invalid entities or tokens. An entity is considered invalid if it does 
not present the common Wikidata pattern: a valid prefix (e.g. wd, wdt, etc) along with a valid 
identifier, which always is composed by a Q and a number (e.g. Q80871). A few queries contained 
invalid numbers such as \textit{“t1231“} which are corrected by removing the \textit{t}. Some 
other queries presented syntax errors or had some clauses missing, which were corrected. Finally, 
a simple canonicalization process was performed over the variable names in order to follow 
similar conventions (e.g. use sbj and obj as base variable names).

Aside from that, some cases were duplicated due to the observation that they contained the same 
normalised question or the same SPARQL query. Those with the same normalised question but 
different SPARQL queries were merged into one case, though just one SPARQL query is used later. 
Other cases with the same SPARQL query (regardless of whether their normalised question is the 
same or not) were discarded.

% TODO: add explanation of complex cases
In summary, from the 30,226 cases contained in the LC-QuAD 2.0 dataset, 2,520 cases were 
discarded: of these, 2,478 cases were discarded for having an invalid question or query, and 42 
cases were duplicate cases. Note that 52\% of this final preprocessed dataset is composed of 
complex questions (a deeper explanation of which cases are considered complex can be found in 
Annex B). After preprocessing, this dataset is used to train the Neural-based models used in 
this work.

\subsection{Query Template Dataset}
\label{cap4:experimentalDesign/QaDataset/queryTemplate}
From the cleaned LC-QuAD 2.0 dataset, a Query Template dataset was created for training the 
Query Template generator model. More specifically, a query template was inferred from each 
SPARQL query by replacing every subject and object entity in the query with a placeholder. In 
order to maintain some part of the semantic meaning of the entity that is being replaced, we 
used different types of placeholder labels. For example, an entity that was the subject of the 
first query triple will be replaced by a placeholder labeled as \textit{“sbj\_1”}. For numerical 
or string values, the placeholder would be \textit{“num”} or \textit{“str\_value”}, respectively. 
This is a different approach to generate query templates than the one followed by 
Yin et al.~\cite{nmt:nl-to-sparql-Yin19}, where instead they replaced entities by one-letter 
names such as A, B or C. 

\begin{lstlisting}[captionpos=b, 
    caption=SPARQL query and its Query Template version., 
    label=lst:queryTemplateExampleLcquad2,
    basicstyle=\ttfamily,frame=single]
# SPARQL Query
SELECT ?obj WHERE { wd:Q14752155 wdt:P19 ?obj }

# Query Template
SELECT ?obj WHERE { <sbj_1> wdt:P19 ?obj }
\end{lstlisting}

An example can be seen in Listing~\ref{lst:queryTemplateExampleLcquad2}, where in the SPARQL 
query for the question \textit{“Where was Pedro Pascal born?”} the entity Q14752155 from Pedro 
Pascal was replaced for the sbj\_1 placeholder. All queries from the cleaned dataset were able 
to be transformed into their query template version.

\subsection{Sequence Labeling Dataset}
\label{cap4:experimentalDesign/QaDataset/queryTemplate}
In order to train the Sequence Labeling model from the Slot Filling module, a dataset was 
needed. As per the Query Template dataset, the Slot Labeling dataset is based on the cleaned 
LC-QuAD 2 dataset. The creation of this dataset is divided into two stages. 

The first stage consists of identifying the entity-label mapping in terms of which part of the 
question corresponds to each entity from the ones contained in the SPARQL query. In the example 
of \textit{“Where was Pedro Pascal born?”}, the label \textit{“Pedro Pascal”} contained in the 
question should correspond to the entity Q14752155 corresponding to the Chilean actor Pedro 
Pascal. 

In the second stage, the labels identified previously are associated with the placeholders used 
for the creation of the Query Template. In this case, the placeholder sbj\_1 shown in Listing X 
was corresponding to the entity Q14752155, therefore the label \textit{“Pedro Pascal”} will be 
associated with the placeholder sbj\_1. We use the BIO label format as the output format for the
Sequence Tagger, as shown in Listing~\ref{lst:sequenceLabelingExampleLcquad2} (the '?' is 
ignored).

\begin{lstlisting}[captionpos=b, 
    caption=BIO label representation for a Natural Language Question., 
    label=lst:sequenceLabelingExampleLcquad2,
    basicstyle=\ttfamily,frame=single]
# NL Question
Where was Pedro Pascal born ?

# Expected label output
O O B-sbj_1 B-sbj_1 O
\end{lstlisting}

Some labels weren’t able to be associated with their corresponding placeholders due to the 
question rephrasing performed over the LC-QuAD. The main issue was the discrepancy between the 
Wikidata label associated with an entity, and its current label in the question (which may be 
modified due to the rephrasing). To compensate for the loss of information we decided to add 
both verbalised and paraphrased valid cases. This gives us 36,854 valid cases from the 52,853 
initial cases (counting for each case its verbalised and paraphrased version), which is almost 
70\% of the dataset.

\subsection{Final Dataset Format}
\label{cap4:experimentalDesign/QaDataset/finalFormat}
After extracting the query templates, the entities, and their entity/placeholder slots, a final 
LC-QuAD 2 dataset was built containing all this information. In summary, each case possesses a 
question identifier, the question string, a Wikidata SPARQL query, their entities with the 
corresponding label, a query template deduced from the SPARQL query, and a mapping between the 
entity labels and the query template placeholders. 

We proposed a normalized format and delivered a dataset that not only can be used to evaluate 
systems on the QAKG task, but also over the tasks of Entity Linking, Name Entity Recognition, 
Slot Filling, Intent Classification, or Semantic Parsing. A sample of the entire dataset can be 
found in Appendix B.

\section{System Implementation}
\label{cap4:experimentalDesign/systemImplementation}
The system implementation was divided into three main modules which represent each stage 
described in the \textit{System Overview} chapter: Query Generation, Entity Linking, and Slot 
Filling. These modules were assembled in our proposed Question Answering system, which receives 
a Natural Language question as an input, and returns a Wikidata SPARQL query that can be executed 
on the Wikidata service endpoint as an output. The Question Answering system only returns a valid 
SPARQL query; otherwise, if one cannot be computed, it will return nothing.

The implementation of the two variants on the Query Generation module (SPARQL Query Generator 
and Query Template Generator) was divided into two stages. First the models were trained on 
Google Colaboratory\footnote{\url{https://colab.research.google.com/}} using the 
ConvS2s~\cite{nmt:convS2S-GehringAGYD17} model implementation included in the Fairseq 
framework\footnote{\url{https://fairseq.readthedocs.io/en/latest/}} and the settings described in the 
Fairseq Model subsection in the \textit{System Overview} chapter (we recall that this model had 
the best performance in the comparison of Yin et al.~\cite{nmt:nl-to-sparql-Yin19}). After the 
training phase, a Fairseq Wrapper was implemented in order to import the models and perform the 
prediction over new data. These wrappers receive the NL question as the input and can return 
either the best predicted sequence, or a list with the best N predictions (where N can be a 
number defined by the user).

The various individual Entity Linking systems (DBpedia Spotlight~\cite{EL:dbpedia-spotlight-MendesJGB11}, 
AIDA~\cite{EL:aida-tool-YosefHBSW11, EL:aida-HoffartYBFPSTTW11}, TAGME~\cite{EL:tagme-FerraginaS10}, 
OpenTapioca~\cite{EL:opentapioca-Delpeuch19}) were implemented using the web services each system 
has available. Then, a Entity Linking wrapper was created for each system, were they received the 
NL question as the input and return a set of annotations which contain a Wikidata entity, the 
label in the question that entity was associated with, and the score the system gives to the 
respective annotations (which varies across systems). Then, the Precision Priority system and 
Voting systems were implemented combining the individual Entity Linking wrappers, maintaining 
the same input–-output frame.

Finally, the Slot Filling module implementation followed similar stages to those for the Query 
Generation module. A Sequence Labeling model is trained using Google Colaboratory and the Flair 
Sequence Tagger from the Flair framework\footnote{\url{https://github.com/flairNLP/flair/}} and 
the setting described in the \textit{System Overview} chapter. Then, a Flair wrapper was 
implemented to import the Sequence Labeling model and perform the tagging over new data. This 
wrapper was used to implement the Slot Filling system, where the input consists of an NL question, 
a SPARQL Template, and the Entity Linking annotations. Then, the output is a SPARQL query built 
from the given set of values, only if the Slot Filling is successful.

The entire system was implemented using Python 3.6. The implementation and the datasets used can 
be found in the following Github repository: \url{https://github.com/ddiomedi2906/Wikidata_QA_Benchmark/}.

\section{Experiments}
\label{cap4:experimentalDesign/experiments}
Several experiments were conducted for the purpose of getting some insights to address the 
proposed research questions. Some experiments were performed to determine the optimal setting of 
the proposed Question Answering (QA) system, in terms of which components are the most 
appropriate to use (e.g. which Entity Linking deliver the best results), while others are done 
to validate the performance of each individual module (Query Generation, Entity Linking, Slot 
Filling) and the end-to-end QA system overall.

\subsection{Datasets}
\label{cap4:experimentalDesign/datasets}
As a summary, the datasets that are used to perform the experiments are the following:

\begin{itemize}
    \item LC-QuAD 2~\cite{dataset:lcquad2-DubeyBA019} (cleaned version), which contains 27,706 
    questions over Wikidata (from the original 30,000) that are divided into a training set and 
    a validation set in a ratio of 80/20. The training set is only used to train the Query 
    Generator models and the Sequence Tagger model. The validation set is used to evaluate the 
    performance of the end-to-end QA system and its modules. We use a split method that 
    guarantees that all the Wikidata properties are included in at least one case in the 
    training set, thus giving the possibility to a Query Template Generator model to learn from 
    all properties contained in the dataset.
    \item QALD-7~\cite{dataset:qald7-UsbeckNHKRN17}, which is divided into a training set (100 
    cases) and a test set (50 cases) for Wikidata. Since neither of these datasets are used for 
    training in our experiments, both splits are merged to be used as one dataset to evaluate 
    the performance of the end-to-end QA system and some of its modules. Note that some LC-QuAD 2 
    templates are based on QALD-7 cases.
    \item WikiSPARQL, which is a novel dataset generated in this work with the objective of 
    measuring how capable our system is to generalize to cases that are not necessarily based on 
    the templates used in the LC-QuAD 2 dataset. This dataset consists of 100 cases of manually 
    created questions over Wikidata, and is used to evaluate the performance of the end-to-end 
    QA system and its modules.
\end{itemize}

An overview of the each datasets used in this work can be see in Table 1.

\begin{table}[h!]
    \centering
    \begin{tabular}{|l|l|l|l|l|}
    \hline
    \textbf{Dataset} & \textbf{Size} & \textbf{Entities} & \textbf{Properties} & \textbf{Type} \\ \hline
    LC-QuAD 2        & 27,706        & 21,151            & 4,447               & Tran/Valid    \\ \hline
    QALD-7           & 150           & 178               & 100                 & Test          \\ \hline
    WikiSPARQL       & 100           & 132               & 86                  & Test          \\ \hline
    \end{tabular}
    \caption{Comparison of Wikidata datasets used for experiments.}
    \label{table:datasetsOverview}
\end{table}

\subsection{Query Template Generation}
\label{cap4:experimentalDesign/queryTemplateGeneration}
The Query Template Generator is trained using the LC-QuAD training set, and is evaluated over 
the training and validation datasets using Perplexity, and BLEU score. Then, the model is 
evaluated over the LC-QuAD validation, QALD-7 and WikiSPARQL datasets using BLEU score and 
string-match accuracy.

The only point of reference for this model is to use the Baseline Query Generator model, and 
evaluate which one of the two better predicts which Query Template corresponds for each case. 
The way this comparison can be done is to take the Baseline SPARQL query output, and convert it 
into its Query Template version by removing the Wikidata entities that the query contains.

\subsection{Entity Linking}
\label{cap4:experimentalDesign/entityLinking}
In order to determine which Entity Linking system performs the best, each Individual Entity 
Linking (IEL) system, along with the Ensemble Entity Linking (EEL) systems proposed in this work, 
are evaluated over a set of datasets and their results are compared. Note that the results over 
the IEL systems are also used to determine the priorities that are used in the EEL systems, as 
explained in the \textit{Theoretical Framework} chapter. Aside from that, in the evaluation of 
the EEL systems proposed (Precision Priority and Voting system), we test the different 
configurations described in the \textit{System Overview} chapter. 

The performance of each Entity Linking system is measured using Precision, Recall and F1-score 
over their output annotations. Both macro and micro measures are used. First, IEL systems are 
evaluated over the LC-QuAD 2 validation set in order to select the priority that each IEL system 
will have for the EEL systems. The priority is determined by comparing the IEL systems’ precision, 
so an IEL system with higher Precision will have higher priority. Then, EEL systems are also 
evaluated over the LC-QuAD 2 validation set. After that, both EEL and IEL are evaluated over 
QALD-7 and our WikiSPARQL datasets.

As a point of reference we calculate the best scenario for an EEL system with an \textit{“oracle”}, 
where given the results from the IEL systems, the EEL system is capable of choosing only the 
correct entities. Note that if there is any correct entity that is not recognized for any of the 
IEL systems, the EEL system itself will not be able to deliver that entity as part of the output 
annotations. This way, we are setting the performance expectation for an EEL system that should 
be able to recognize which of the entities delivered are part of the correct set of entities. We 
will refer to this best EEL system scenario, as the Oracle Entity Linking system (OEL).


\subsection{Sequence Labeling and Slot Filling}
\label{cap4:experimentalDesign/seqLabAndSlotFilling}
The Slot Filling system is evaluated in two phases: evaluation over the Sequence Tagger 
performance, and evaluation over the Slot Filling algorithm. 

First, the Sequence Tagger model is trained using the LC-QuAD 2 training set, and is evaluated 
using Precision, Recall, and F1-score over the output labels. Then, the model is evaluated over 
the LC-QuAD 2 validation, QALD-7 and WikiSPARQL datasets using the same metrics.

After that, the Slot Filling system is evaluated in terms of how the filling process is performed. 
Then, using the results obtained from the \textit{Query Template Generation} and the 
\textit{Entity Linking} experiments, we evaluate the ratio of correct/incorrect cases given the 
different input cases (e.g. all the incoming input is correct, some of the systems delivered an 
incorrect output, and so on).

\subsection{Question Answering over Knowledge Graphs}
\label{cap4:experimentalDesign/QAKG}
The evaluation of the overall QA system is performed using two approaches: the traditional 
approach of evaluating answers given by the system~\cite{qa:qald-Lopezetal2013}, and (as used by 
many Neural-based systems) the approach of evaluating the SPARQL query 
returned~\cite{nmt:nl-to-sparql-Yin19}. The baseline for both approaches is the SPARQL Query 
generator based on the ConvS2S model that performs the best among the Neural-based models 
according to the Ying et al.’s work~\cite{nmt:nl-to-sparql-Yin19}.

The first approach consists of evaluating the performance of the QA system using Precision, 
Recall and F1-score over the system’s answers. This approach values more the final output of the 
system, regardless of how it was retrieved. The second approach consists of evaluating the 
performance of the QA system using BLEU score and string-match accuracy. This last approach 
values more the query string used to generate the answer, and how close it is for the expected 
query. Both approaches are evaluated over the LC-QuAD 2 validation set, QALD-7, and WikiSPARQL.

Aside from a general performance evaluation, we study the performance of the system considering 
not only the best SPARQL query answer, but the five SPARQL queries with the highest scores given 
by the system. This can be performed given that the Query Template Generator is capable of 
returning a list of templates. We believe that this is an interesting analysis given that a 
possible heuristic to improve the system performance is to look for the correct answer by trying 
more than one possible SPARQL query, which would work particularly well under the assumption 
that the input question indeed has some answer, and that many of the incorrect SPARQL query 
outputs will not generate answers. This analysis is evaluated over the LC-QuAD 2 validation set, 
QALD-7, and WikiSPARQL.

Then, in order to understand which components of a question and its SPARQL query influence the 
QA system performance more, an evaluation per template is performed. The same evaluation over 
the entire dataset is split into partitions corresponding to each LC-QuAD 2 base template. Given 
that only LC-QuAD 2 is a template-based dataset, this analysis is only performed over that 
dataset.

Finally, a more granular analysis is conducted in order to understand how much each stage 
influences the overall results. We compare the correct and incorrect cases of each module (Query 
Template generation, Entity Linking, and Slot Filling) with the correct/incorrect cases of the 
QA system, and calculate the ratio of error of each module and how much each one of them 
contributes to the overall error the QA system has. This analysis is evaluated over the LC-QuAD 2 
validation set, QALD-7, and WikiSPARQL.
